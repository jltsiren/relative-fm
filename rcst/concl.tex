

\section{Discussion}\label{section:discussion}

We have introduced relative suffix trees (\RCST), a new kind of compressed suffix tree for repetitive sequence collections. Our \RCST{} compresses the suffix tree of an individual sequence relative to the suffix tree of a reference sequence. It combines an already known relative suffix array with a novel relative-compressed longest common prefix representation (\RLCP). When the sequences are similar enough (e.g., two human genomes), the \RCST{} requires about 3 bits per symbol on each target sequence. This is close to the space used by the most space-efficient compressed suffix trees designed to store repetitive collections in a single tree, but the \RCST{} provides a different functionality as it indexes each sequence individually. The \RCST{} supports query and navigation operations within a few microseconds, which is competitive with the largest and fastest compressed suffix trees.

While our \RCST{} implementation provides competitive time/space trade-offs, there is still much room for improvement. Most importantly, some of the construction algorithms require significant amounts of time and memory. In many places, we have chosen simple and fast implementation options, even though there could be alternatives that require significantly less space without being too much slower.

Our \RCST{} is a relative version of the \CSTnpr. Another alternative for future work is a relative \CSTsada, using \RLZ{} compressed bitvectors for suffix tree topology and \PLCP. %Based on our preliminary experiments, the main obstacle is the compression of phrase pointers. Relative pointers work well when most differences between the reference and the target are single-character substitutions. As suffix sorting multiplies the differences and transforms substitutions into insertions and deletions, we need new compression schemes for the pointers.

