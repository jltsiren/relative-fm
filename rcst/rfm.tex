

\section{Relative FM-index}

The \emph{relative FM-index} (\RFM) \cite{Belazzougui2014} is a compressed
suffix array of a sequence relative to the \CSA{} of another sequence.
The index is based on approximating the
\emph{longest common subsequence} (\LCS) of $\mBWT_{R}$ and $\mBWT_{S}$,
where $R$ is the reference sequence and $S$ is the target sequence, and
storing several structures based on the common subsequence. Given a
representation of $\mBWT_{R}$ supporting \rank{} and \select{}, we can use the
relative index $\mRFM_{S \mid R}$ to simulate \rank{} and \select{} on
$\mBWT_{S}$.

\subsection{Basic index}

Assume that we have found a long common subsequence of sequences $X$ and $Y$.
We call positions $X[i]$ and $Y[j]$ \emph{lcs-positions}, if they are in the
common subsequence. If $B_{X}$ and $B_{Y}$ are the binary sequences marking
the common subsequence ($X[\select_1(B_{X},i)] = Y[\select_1(B_{Y},i)]$), we
can move between lcs-positions in the two sequences with \rank{} and \select{}
operations. If $X[i]$ is an lcs-position, the corresponding position in
sequence $Y$ is $Y[\mselect_{1}(B_{Y}, \mrank_{1}(B_{X}, i))]$. We denote this
pair of \emph{lcs-bitvectors} $\mAli(X,Y) = \langle B_X,B_Y \rangle$.

In its most basic form, the relative FM-index $\mRFM_{S \mid R}$ only supports
\find{} queries by simulating \rank{} queries on $\mBWT_{S}$. It does this by
storing $\mAli(\BWT_{R},\BWT_{S})$ and the complements $\mCS(\mBWT_{R})$ and
$\mCS(\mBWT_{S})$ of the common subsequence. The lcs-bitvectors are compressed
using \emph{entropy-based compression} \cite{Raman2007}, while the complements
are stored in structures similar to the reference $\mBWT_{R}$.

To compute $\mrank_{c}(\mBWT_{S}, i)$, we first determine the number of
lcs-positions in $\mBWT_{S}$ up to position $S[i]$ with $k =
\mrank_{1}(B_{\mBWT_{S}}, i)$. Then we find the lcs-position $k$ in $\mBWT_{R}$
with $j = \mselect_{1}(B_{\mBWT_{R}}, k)$. With these positions, we can compute
\begin{eqnarray*}
\mrank_{c}(\mBWT_{S}, i) &=& \mrank_{c}(\mBWT_{R}, j) \\
   && - \mrank_{c}(\mCS(\mBWT_{R}),j-k) \\
   && + \mrank_{c}(\mCS(\mBWT_{S}), i-k).
\end{eqnarray*}

\subsection{Relative select}

We can implement the entire functionality of a compressed suffix array with
\rank{} queries on the \BWT. However, if we use the \CSA{} in a compressed
suffix tree, we also need \select{} queries to support \emph{forward
searching} with $\mPsi$ and $\mChild$ queries. We can always implement
\select{} queries by binary searching with \rank{} queries, but the result
will be much slower than the \rank{} queries.

A faster alternative to support \select{} queries in the relative FM-index
is to build a \emph{relative select} structure \rselect{}~\cite{Boucher2015}.
Let $\mF_{X}$ be a sequence consisting of the characters of sequence $X$ in
sorted order. Alternatively, $\mF_{X}$ is a sequence such that $\mF_{X}[i] =
\mBWT_{X}[\mPsi_X(i)]$. The relative select structure consists of bitvectors
$\mAli(\mF_{R}, \mF_{S})$, where $B_{\mF_{R}}[i] = B_{\mBWT_{R}}[\mPsi_R(i)]$ 
and $B_{\mF_{S}}[i] = B_{\mBWT_{S}}[\mPsi_S(i)]$, as well as the \C{} array
$\mC_{\mLCS}$ for the common subsequence.

To compute $\mselect_{c}(\mBWT_{S}, i)$, we first determine how many of
the first $i$ occurrences of character $c$ are lcs-positions with $k =
\mrank_{1}(B_{\mF_{S}}, \mC_{\mBWT_{S}}[c] + i) - \mC_{\mLCS}[c]$. Then we check
from bit $B_{\mF_{S}}[\mC_{\mBWT_{S}}[c] + i]$ whether the occurrence we are
looking for is an lcs-position or not. If it is,
we find the position in $\mBWT_{R}$ as $j = \mselect_{c}(\mBWT_{R},
\mselect_{1}(B_{\mF_{R}}, \mC_{\mLCS}[c] + k)- \mC_{R}[c])$, and then map $j$ to
$\mselect_{c}(\mBWT_{S}, i)$ by using $\mAli(\mBWT_{R}, \mBWT_{S})$. Otherwise we
find the occurrence in $\mCS(\mBWT_{S})$ with $j = \mselect_{c}(\mCS(\mBWT_{S}),
i-k)$, and return $\mselect_{c}(\mBWT_{S}, i) = \mselect_{0}(B_{\mBWT_{S}}, j)$.

\subsection{Full functionality}

If we want the relative FM-index to support \locate{} and \extract{} queries,
we cannot build it from any common subsequence of $\mBWT_{R}$ and $\mBWT_{S}$.
We need a \emph{bwt-invariant subsequence} \cite{Belazzougui2014}, where the
alignment of the \BWT{}s is also an alignment of the original sequences.

\begin{definition}\label{def:bwt-invariant}
Let $X$ be a common subsequence of $\mBWT_{R}$ and $\mBWT_{S}$, and let
$\mBWT_{R}[i_{R}]$ and $\mBWT_{S}[i_{S}]$ be the lcs-positions corresponding to
$X[i]$. Subsequence X is bwt-invariant if
$$
\mSA_{R}[i_{R}] < \mSA_{R}[j_{R}] \iff \mSA_{S}[i_{S}] < \mSA_{S}[j_{S}]
$$
for all positions $i, j \in \set{1, \dotsc, \abs{X}}$.
\end{definition}

In addition to the structures already mentioned, the full relative FM-index
has another pair of lcs-bitvectors, $\mAli(R,S)$, which marks the
bwt-invariant subsequence in the original sequences. If $\mBWT_{R}[i_{R}]$ and
$\mBWT_{S}[i_{S}]$ are lcs-positions, we set $B_{R}[\mSA_{R}[i_{R}]-1] = 1$ and
$B_{S}[\mSA_{S}[i_{S}]-1] = 1$.\footnote{For simplicity, we assume that the
endmarker is not a part of the bwt-invariant subsequence. Hence $\mSA[i] > 1$
for all lcs-positions $\mBWT[i]$.}

To compute the answer to a $\mlocate(i)$ query, we start by iterating
$\mBWT_{S}$ backwards with \LF{} queries, until we find an lcs-position
$\mBWT_{S}[i']$ after $k$ steps. Then we map position $i'$ to the corresponding
position $j'$ in $\mBWT_{R}$ by using $\mAli(\mBWT_{R},\mBWT_{S})$. Finally we
determine $\mSA_{R}[j']$ with a \locate{} query in the reference index, and map
the result to $\mSA_{S}[i']$ by using $\mAli(R,S)$.\footnote{If $\mBWT_{S}[i']$
and $\mBWT_{R}[j']$ are lcs-positions, the corresponding lcs-positions in the
original sequences are $S[\mSA_{S}[i']-1]$ and $R[\mSA_{R}[j']-1]$.} The result
of the $\mlocate(i)$ query is $\mSA_{S}[i']+k$.

The $\mISA_{S}[i]$ access required for \extract{} queries is supported in a
similar way. We find the lcs-position $S[i+k]$ for the smallest $k \ge 0$, and
map it to the corresponding position $R[j]$ by using $\mAli(R,S)$. Then we
determine $\mISA_{R}[j+1]$ by using the reference index, and map it back to
$\mISA_{S}[i+k+1]$ with $\mAli(\mBWT_{R},\mBWT_{S})$. Finally we iterate
$\mBWT_{S}$ $k+1$ steps backward with \LF{} queries to find $\mISA_{S}[i]$.

If the target sequence contains long
insertions not present in the reference, we may also want to include
some \SA{} and \ISA{} samples for querying those regions.

\subsection{Finding a bwt-invariant subsequence}

With the basic relative FM-index, we approximate the longest common
subsequence of $\mBWT_{R}$ and $\mBWT_{S}$ by partitioning the \BWT{}s according
to lexicographic contexts, finding the longest common subsequence for each
pair of substrings in the partitioning, and concatenating the results. The
algorithm is fast, easy to parallelize, and quite space-efficient. As such,
\RFM{} construction is practical, having been tested with datasets of hundreds
of gigabytes in size.

To find a bwt-invariant subsequence, we start by \emph{matching} each suffix
of the reference sequence with the lexicographically nearest suffixes of the
target sequence. In contrast to the original algorithm \cite{Belazzougui2014}, we
only match suffixes that are lexicographically adjacent in the \emph{mutual
suffix array} of the two sequences.

\begin{definition}
Let $R$ and $S$ be two sequences, and let $\mSA = \mSA_{RS}$ and $\mISA =
\mISA_{RS}$. The \emph{left match} of suffix $R[i,\abs{R}]$ is the suffix
$S[\mSA[\mISA[i]-1] - \abs{R}, \abs{S}]$, if $\mISA[i] > 1$ and
$\mSA[\mISA[i]-1]$ points to a suffix of $S$ ($\mSA[\mISA[i]-1] > \abs{R}$).
The \emph{right match} of suffix $R[i,\abs{R}]$ is the suffix
$S[\mSA[\mISA[i]+1] - \abs{R}, \abs{S}]$, if $\mISA[i] < \abs{RS}$ and
$\mSA[\mISA[i]+1]$ points to a suffix of $S$.
\end{definition}

Instead of using the mutual suffix array, we can use $\mCSA_{R}$, $\mCSA_{S}$,
and the \emph{merging bitvector} $B_{R,S}$ of length $\abs{RS}$. We set
$B_{R,S}[i] = 1$, if $\mSA_{RS}[i]$ points to a suffix of $S$. We can build the
merging bitvector in $\Oh(\abs{S} \cdot t_{\mLF})$ time, where $t_{\mLF}$ is
the time required for an \LF{} query, by extracting $S$ from $\mCSA_{S}$ and
backward searching for it in $\mCSA_{R}$ \cite{Siren2009}. Suffix
$R[i,\abs{R}]$ has a left (right) match, if $B_{R,S}[\mselect_{0}(B_{R,S},
\mISA_{R}[i])-1] = 1$ ($B_{R,S}[\mselect_{0}(B_{R,S}, \mISA_{R}[i])+1] = 1)$).

Our next step is building the \emph{match arrays} $\mleft$ and $\mright$,
which correspond to the arrays $A[\cdot][2]$ and $A[\cdot][1]$ in the original
algorithm. This is done by traversing $\mCSA_{R}$ backwards from
$\mISA_{R}[\abs{R}] = 1$ with \LF{} queries and following the left and the
right matches of the current suffix. During the traversal, we maintain
the invariant $j = \mSA_{R}[i]$ with $(i,j) \leftarrow (\mLF_{R}(i), j-1)$. If
suffix $R[j,\abs{R}]$ has a left (right) match, we use the shorthand $l(j) =
\mrank_{1}(B_{R,S}, \mselect_{0}(B_{R,S}, i)-1)$ ($r(j) = \mrank_{1}(B_{R,S},
\mselect_{0}(B_{R,S}, i)+1)$) to refer to its position in $\mCSA_{S}$.

We say that suffixes $R[j,\abs{R}]$ and $R[j+1,\abs{R}]$ have the same left
match if $l(j) = \mLF_{S}(l(j+1))$. Let $R[j,\abs{R}]$ to $R[j+\ell,\abs{R}]$
be a maximal run of suffixes having the same left match, with suffixes
$R[j,\abs{R}]$ to $R[j+\ell-1,\abs{R}]$ starting with the same characters as
their left matches.\footnote{The first character of a suffix can be determined
by using the $\mC$ array.} We find the left match of suffix $R[j,\abs{R}]$ as
$j' = \mSA_{S}[l(j)]$ by using $\mCSA_{S}$, and set $\mleft[j,j+\ell-1] =
[j',j'+\ell-1]$. The right match array $\mright$ is built in a similar way.

The match arrays require $2\abs{R} \log \abs{S}$ bits of space. If sequences
$R$ and $S$ are similar, the runs in the arrays tend to be long. Hence we can
run-length encode the match arrays to save space. The traversal takes
$\Oh(\abs{R} \cdot (t_{\mLF} + t_{\mrank} + t_{\mselect}) + rd \cdot
t_{\mLF})$ time, where $t_{\mrank}$ and $t_{\mselect}$ denote the time
required by \rank{} and \select{} operations, $r$ is the number of runs in the
two arrays, and $d$ is the suffix array sample interval in
$\mCSA_{S}$.\footnote{The time bound assumes text-order sampling.}

The final step is finding the longest increasing subsequence $X$ of arrays
$\mleft$ and $\mright$, which corresponds to a common subsequence of $R$ and
$S$. More precisely, we want to find a binary sequence $B_{R}[1,\abs{R}]$,
which marks the common subsequence in $R$, and an integer sequence $X$, which
contains the positions of the common subsequence in $S$. The goal is to make
sequence $X$ strictly increasing and as long as possible, with
$X[\mrank_{1}(B_{R}, i)]$ being either $\mleft[i]$ or $\mright[i]$. This can
be done in $\Oh(\abs{R} \log \abs{R})$ time with $\Oh(\abs{R} \log \abs{R})$
bits of additional working space with a straightforward modification of the
dynamic programming algorithm for finding the longest increasing subsequence.
While the dynamic programming tables can be run-length encoded, the time and
space savings are negligible or even non-existent in practice.

As sequence $X$ is strictly increasing, we can convert it into binary sequence
$B_{S}[1,\abs{S}]$, marking the values in sequence $X$ with \onebit{}s.
Afterwards, we can consider the binary sequences $B_{R}$ and $B_{S}$ as the
lcs-bitvectors $\mAli(R,S)$. Because every suffix of $R$ starts with the same
character as its matches stored in the $\mleft$ and $\mright$ arrays,
subsequences $R[B_{R}]$ and $S[B_{S}]$ are identical. As each suffix
$R[i,\abs{R}]$ with $B_{R}[i] = 1$ is paired with its left match or right
match in sequence $S$, no other suffix of $R$ or $S$ is lexicographically
between the two paired suffixes.

For any $i$, let $i_{R} = \mselect_{1}(B_{R}, i)$ and $i_{S} =
\mselect_{1}(B_{S}, i)$ be the lcs-positions of rank $i$. Then,
$$
\mISA_{R}[i_{R}] < \mISA_{R}[j_{R}] \iff \mISA_{S}[i_{S}] < \mISA_{S}[j_{S}]
$$
for $1 \le i,j \le \abs{X}$, which is equivalent to the condition in
Definition~\ref{def:bwt-invariant}. We can convert $\mAli(R,S)$ to
$\mAli(\mBWT_{R},\mBWT_{S})$ in $\Oh((\abs{R}+\abs{S}) \cdot t_{\mLF})$ time by
traversing $\mCSA_{R}$ and $\mCSA_{S}$ backwards. The resulting subsequence of
$\mBWT_{R}$ and $\mBWT_{S}$ is bwt-invariant.

Note that the full relative FM-index is more limited than the basic index,
because it does not handle \emph{substring moves} very well. Let $R = xy$ and
$S = yx$, for two random sequences $x$ and $y$ of length $n/2$ each. Because
$\mBWT_{R}$ and $\mBWT_{S}$ are very similar, we can expect to find a common
subsequence of length almost $n$. On the other hand, the length of the longest
bwt-invariant subsequence is around $n/2$, because we can either match the
suffixes of $x$ or the suffixes of $y$ in $R$ and $S$, but not both.

