

\section{Introduction}

The \emph{suffix tree} \cite{Weiner1973} is one of the most powerful bioinformatic tools to
answer complex queries on DNA and protein sequences \cite{Gus97,Ohl13,MBCT15}.
A serious problem that hampers its wider use on large genome sequences is its
size, which may be 10--20 bytes per character. In addition, the non-local
access patterns required by most interesting problems solved with suffix trees
complicate secondary-memory deployments. This problem has led to numerous
efforts to reduce the size of suffix trees by representing them using 
\emph{compressed data structures} \cite{Sadakane2007,Fischer2009a,Ohlebusch2009,Ohlebusch2010,Fis10,Russo2011,Gog2011a,GO13b,Abeliuk2013,Navarro2014a,Navarro2015,Ock15,BCGPR15}, 
leading to \emph{compressed suffix trees} (\CST). Currently, the smallest
\CST{} is the so-called \emph{fully-compressed suffix tree} (\FCST)
\cite{Russo2011,Navarro2014a}, which uses 5 \emph{bits} per character (bpc)
for DNA sequences, but takes milliseconds to simulate suffix
tree navigation operations. In the other extreme, Sadakane's \CST{}
\cite{Sadakane2007,Gog2011a} uses about 12~bpc and operates in
microseconds, and even nanoseconds for the simplest operations.

A space usage of 12~bpc may seem reasonable to handle, for example, one human
genome, which has about 3.1 billion bases: it can be operated within a
RAM of 4.5~GB (the representation contains the sequence as well). However,
as the price of sequencing has fallen, sequencing the genomes of a large
number of individuals has become a routine activity. The \emph{1000 Genomes
Project} \cite{1000GP2015} sequenced the genomes of several thousand humans,
while newer projects can be orders of magnitude larger. This has made the
development of techniques for storing and analyzing huge amounts of sequence
data flourish.

Just storing 1000 human genomes using a 12~bpc \CST{} requires almost 4.5~TB, which
is much more than the amount of memory available in a commodity server. Assuming that
a single server has 256 GB of memory, we would need a cluster of 18 servers to
handle such a collection of \CST{}s (compared to over 100 with classical suffix
tree implementations!). With the smaller (and much slower) \FCST, this would
drop to 7--8 servers. It is clear that further space reductions in the
representation of compressed suffix trees would lead to reductions in hardware, communication,
and energy costs when implementing complex searches over large genomic
databases.

An important characteristic of those large genome databases is that they
usually consist of the genomes of individuals of the same or closely related
species. This implies that the collections are highly \emph{repetitive}, that
is, each genome can be obtained by concatenating a relatively small number of
substrings of other genomes
and adding a few new characters. When repetitiveness is considered, much higher
compression rates can be obtained in compressed suffix trees. For example, it is possible to reduce
the space to 1--2~bpc (albeit with operation times in the milliseconds)
\cite{Abeliuk2013}, or to 2--3~bpc with operation times in the microseconds
\cite{Navarro2015}. Using 2~bpc, our 1000 genomes could be handled
with just 3 servers with 256~GB of memory. We note, however, that these \CST{}s index
the whole collection and not individual sequences, which makes a difference in
the types of queries that can be answered. This also makes a distributed
implementation less obviously scalable.

Compression algorithms best capture repetitiveness by using \emph{grammar-based}
compression or \emph{Lempel-Ziv} compression.\footnote{We refer to ``long-range''
repetitiveness, where similar texts may be found far away in the text
collection.} In the first case \cite{KY00,CLLPPSS05} one finds a context-free
grammar that generates (only) the text collection. The more repetitive the
collection is, the smaller the grammar becomes. Rather than compressing the text directly,
the current \CST{}s for repetitive collections \cite{Abeliuk2013,Navarro2015}
apply grammar-based compression on the data structures that simulate the suffix tree.
Grammar-based compression yields relatively easy direct access to the compressed
sequence \cite{BLRSRW15}, which makes it attractive compared to Lempel-Ziv
compression \cite{ZL77}, despite the latter generally using less space.

Lempel-Ziv compression cuts the collection into \emph{phrases}, each of which
has already appeared earlier in the collection. To extract the content of a phrase, one may have
to recursively extract the content at that earlier position, following a 
possibly long chain of indirections.
So far, the indexes built on Lempel-Ziv compression \cite{KN13} or on
combinations of Lempel-Ziv and grammar-based compression \cite{GGKNP12,GGKNP14,GP15}
support only pattern matching, which is just one of the wide range of
functionalities offered by suffix trees. The high cost to access the data
at random positions lies at the heart of the research on indexes built on
Lempel-Ziv compression.

A simple way out of this limitation is the so-called \emph{relative Lempel-Ziv}
(\RLZ) compression \cite{Kuruppu2010}, where one of the sequences is represented
in plain form and the others can only take phrases from that \emph{reference
sequence}. This enables immediate access for the symbols inside any copied
phrase (as no transitive referencing exists) and, at least if a good reference
sequence has been found, offers compression competitive with the
classical Lempel-Ziv. In our case, taking any random genome per species as the
reference is good enough; more sophisticated techniques have been studied
\cite{KPZ11,KBSCZ12,LPMW16}. Structures for direct access \cite{DG11,Ferrada2014}
and even for pattern matching \cite{DJSS14,Belazzougui2014} have been developed
on top of \RLZ.

In this paper, we develop a \CST{} by augmenting the \emph{relative FM-index}
\cite{Belazzougui2014} with structures based on \RLZ.
On a collection of human genomes, we achieve less than 3~bpc and operate
within microseconds. This performance is comparable to that of a previous \CST{}
for this scenario \cite{Navarro2015}, but our \CST{}s have a different
functionality. We have a separate \CST{} for each sequence, instead of a single
\CST{} for their concatenation. Depending on the application, one kind of \CST{} or
the other is necessary.

Our compressed suffix tree, called \RCST, follows a trend of \CST{}s
\cite{Fischer2009a,Ohlebusch2009,Fis10,Ohlebusch2010,Gog2011a,Abeliuk2013} that use only a pattern-matching index
(called \emph{suffix array}) and an array with the length of the longest common prefix
between each suffix and the previous one in lexicographic order (called \LCP).
We use the relative FM-index as our suffix array, and
compress \LCP{} using \RLZ. On top of the \RLZ{} phrases we build a tree
of range minima that enables fast range minimum queries, as well as
next- and previous-smaller-value queries, on \LCP{} \cite{Abeliuk2013}. All the \CST{} functionality
is built on those queries \cite{Fischer2009a}. Our main algorithmic contribution
is this \RLZ\nobreakdash-based representation of the \LCP{} array with the required extra
functionality.

Another approach to compressing a repetitive collection while supporting interesting queries is to build an automaton that accepts the sequences in the collection, and then index the state diagram as an directed acyclic graph (DAG); see, for example,~\cite{MaciucaEtAl16,PatenEtAl17,Siren17} for recent discussions.  The first data structure to take this approach was the Generalized Compressed Suffix Array (GCSA)~\cite{SVM14,Siren17}, which was designed for pangenomics so queries can return information about sequences not in the collection but that can be obtained from those in the collection by recombination.  The FM-index of an alignment (FMA)~\cite{NaKPLLMP16,NaEtAl17} is similar to the GCSA but indexes only the sequences in the collection: whereas the GCSA conceptually embeds the automaton in a de Bruijn graph, the FMA embeds it in a coloured de Bruijn graph~\cite{IqbalEtAl12}, preserving its specificity.  Both the GCSA and FMA are practical but neither support the full functionality of a suffix tree.  The precursor to the FMA, the suffix tree of an alignment (STA)~\cite{NPCHIMP13,NPLHLMP13}, allows certain disjunctions in the suffix tree's edge labels in order to reduce the size of the tree while maintaining its functionality.  It differs from our approach because it is a single tree for the whole collection and not a separate one for each sequence; also, unlike the FMA, the STA has not been implemented.  Both the STA and FMA divide the sequences in the collection into regions of variation and conserved regions, and depend on the conserved regions being long enough that they can be distinguished from each other and the variations.  This dependency makes these structures vulnerable to even a small change in even one sequence to an otherwise-conserved region, which could hamper their scalability.

