
\documentclass[a4paper,11pt]{llncs}
\usepackage{fullpage}
%\documentclass[10pt]{llncs}

\usepackage{array}
\usepackage{graphicx}
\usepackage{multirow}
\usepackage{textgreek}
\usepackage{url}

\usepackage[utf8]{inputenc}

\usepackage{amsmath}
\usepackage{amssymb}


% Mathematics
\newcommand{\set}[1]{\ensuremath{\{ #1 \}}}
\newcommand{\abs}[1]{\ensuremath{\lvert #1 \rvert}}
\renewcommand{\complement}[1]{\ensuremath{\overline{ #1 }}}

% Suffix trees
\newcommand{\ST}{\textsf{ST}}
\newcommand{\CST}{\textsf{CST}}
\newcommand{\CSTsada}{\textsf{CST\nobreakdash-Sada}}
\newcommand{\GCT}{\textsf{GCT}}
\newcommand{\FCST}{\textsf{FCST}}
\newcommand{\CSTnpr}{\textsf{CST\nobreakdash-NPR}}
\newcommand{\RCST}{\textsf{RCST}}

% Suffix arrays etc.
\newcommand{\SA}{\textsf{SA}}
\newcommand{\ISA}{\textsf{ISA}}
\newcommand{\BWT}{\textsf{BWT}}
\newcommand{\CSA}{\textsf{CSA}}
\newcommand{\FMI}{\textsf{FMI}}
\newcommand{\SSA}{\textsf{SSA}}
\newcommand{\CSAsada}{\textsf{CSA-Sada}}
\newcommand{\RFM}{\textsf{RFM}}
\newcommand{\mSA}{\ensuremath{\mathsf{SA}}}
\newcommand{\mISA}{\ensuremath{\mathsf{ISA}}}
\newcommand{\mBWT}{\ensuremath{\mathsf{BWT}}}
\newcommand{\mF}{\ensuremath{\mathsf{F}}}
\newcommand{\mCSA}{\ensuremath{\mathsf{CSA}}}
\newcommand{\mRFM}{\ensuremath{\mathsf{RFM}}}

% LCP arrays
\newcommand{\LCP}{\textsf{LCP}}
\newcommand{\DLCP}{\textsf{DLCP}}
\newcommand{\PLCP}{\textsf{PLCP}}
\newcommand{\RLCP}{\textsf{RLCP}}
\newcommand{\LCPbyte}{\textsf{LCP\nobreakdash-byte}}
\newcommand{\LCPdac}{\textsf{LCP\nobreakdash-dac}}
\newcommand{\mLCP}{\ensuremath{\mathsf{LCP}}}
\newcommand{\mDLCP}{\ensuremath{\mathsf{DLCP}}}
\newcommand{\mPLCP}{\ensuremath{\mathsf{PLCP}}}
\newcommand{\mRLCP}{\ensuremath{\mathsf{RLCP}}}

% Other structures
\newcommand{\WT}{\textsf{WT}}
\newcommand{\mWT}{\ensuremath{\mathsf{WT}}}
\newcommand{\C}{\textsf{C}}
\newcommand{\mC}{\ensuremath{\mathsf{C}}}
\newcommand{\RLZ}{\textsf{RLZ}}
\newcommand{\mRLZ}{\ensuremath{\mathsf{RLZ}}}
\newcommand{\LCS}{\textsf{LCS}}
\newcommand{\mLCS}{\ensuremath{\mathsf{LCS}}}
\newcommand{\mCS}{\ensuremath{\complement{\mathsf{LCS}}}}
\newcommand{\mleft}{\ensuremath{\mathsf{left}}}
\newcommand{\mright}{\ensuremath{\mathsf{right}}}
\newcommand{\sdarray}{\textsf{sdarray}}
\newcommand{\slarray}{\textsf{slarray}}
\newcommand{\rselect}{\textsf{rselect}}

% Queries
\newcommand{\LF}{\textsf{LF}}
\newcommand{\Psiop}{\textsf{\textPsi}}
\newcommand{\find}{\textsf{find}}
\newcommand{\locate}{\textsf{locate}}
\newcommand{\extract}{\textsf{extract}}
\newcommand{\rank}{\textsf{rank}}
\newcommand{\select}{\textsf{select}}
\newcommand{\nsv}{\textsf{nsv}}
\newcommand{\nsev}{\textsf{nsev}}
\newcommand{\psv}{\textsf{psv}}
\newcommand{\psev}{\textsf{psev}}
\newcommand{\rmq}{\textsf{rmq}}

% Operators
\newcommand{\mLF}{\ensuremath{\mathsf{LF}}}
\newcommand{\mPsi}{\ensuremath{\mathsf{\Psi}}}
\newcommand{\mfind}{\ensuremath{\mathsf{find}}}
\newcommand{\mlocate}{\ensuremath{\mathsf{locate}}}
\newcommand{\mextract}{\ensuremath{\mathsf{extract}}}
\newcommand{\mrank}{\ensuremath{\mathsf{rank}}}
\newcommand{\mselect}{\ensuremath{\mathsf{select}}}
\newcommand{\mlcp}{\ensuremath{\mathsf{lcp}}}
\newcommand{\mpsv}{\ensuremath{\mathsf{psv}}}
\newcommand{\mpsev}{\ensuremath{\mathsf{psev}}}
\newcommand{\mnsv}{\ensuremath{\mathsf{nsv}}}
\newcommand{\mnsev}{\ensuremath{\mathsf{nsev}}}
\newcommand{\mrmq}{\ensuremath{\mathsf{rmq}}}
\newcommand{\Oh}{\ensuremath{\mathsf{O}}}
\newcommand{\oh}{\ensuremath{\mathsf{o}}}
\newcommand{\Th}{\ensuremath{\mathsf{\Theta}}}

% CST operations
\newcommand{\mRoot}{\ensuremath{\mathsf{Root}}}
\newcommand{\mLeaf}{\ensuremath{\mathsf{Leaf}}}
\newcommand{\mAncestor}{\ensuremath{\mathsf{Ancestor}}}
\newcommand{\mCount}{\ensuremath{\mathsf{Count}}}
\newcommand{\mLocate}{\ensuremath{\mathsf{Locate}}}
\newcommand{\mParent}{\ensuremath{\mathsf{Parent}}}
\newcommand{\mFChild}{\ensuremath{\mathsf{FChild}}}
\newcommand{\mNSibling}{\ensuremath{\mathsf{NSibling}}}
\newcommand{\mLCA}{\ensuremath{\mathsf{LCA}}}
\newcommand{\mSDepth}{\ensuremath{\mathsf{SDepth}}}
\newcommand{\mTDepth}{\ensuremath{\mathsf{TDepth}}}
\newcommand{\mLAQ}{\ensuremath{\mathsf{LAQ}}}
\newcommand{\mSLink}{\ensuremath{\mathsf{SLink}}}
\newcommand{\mChild}{\ensuremath{\mathsf{Child}}}
\newcommand{\mLetter}{\ensuremath{\mathsf{Letter}}}

% Other
\newcommand{\onebit}{$1$\nobreakdash-bit}
\newcommand{\zerobit}{$0$\nobreakdash-bit}
\newcommand{\mus}{\textmu{}s}


\title{Relative Compressed Suffix Trees\thanks{This work is funded in part by:
by Fondecyt Project 1-140796; Basal Funds FB0001, Conicyt, Chile;
by Academy of Finland grants 258308 and 250345 (CoECGR); and by the Wellcome
Trust grant [098051].}}

\author{
Travis Gagie\inst{1}
\and
Gonzalo Navarro\inst{2} %\fnmsep
%\thanks{Funded in part by Fondecyt Project 1-140796, Chile, and Basal Funds
%FB0001, Conicyt, Chile.}
\and
Simon J. Puglisi\inst{1}
\and
Jouni Sir\'en\inst{3} %\fnmsep
%\thanks{Funded by the Jenny and Antti Wihuri Foundation, Finland, and Basal
%Funds FB0001, Conicyt, Chile.}
}

\institute{
    Department of Computer Science,
    University of Helsinki, Finland\\
    \email{\{gagie,puglisi\}@cs.helsinki.fi}\\[1ex]
\and
    Center for Biotechnology and Bioengineering, Department of Computer
Science,
    University of Chile, Chile\\
    \email{gnavarro@dcc.uchile.cl}\\[1ex]
\and
    Wellcome Trust Sanger Institute, United Kingdom\\
    \email{jouni.siren@sanger.ac.uk}\\[1ex]
}

\date{}


\pagestyle{plain}

\begin{document}

\maketitle

%\thispagestyle{empty} %for ALENEX
%\setcounter{page}{0} %for ALENEX

\begin{abstract}
\iffalse
This work investigates the use of mutual information between data structures
for similar
datasets to represent the structures in less space. If two data structures are
similar to each
other, one of them can probably be represented by its differences to the
other, while still
supporting efficient queries. Such relative data structures may find use in
bioinformatics,
where the genomes of individuals of the same species are very similar to each
other.
More formally, assume that we have similar datasets R and S. If we build data
structure D for the datasets, we will likely see that D(R) and D(S) have low
relative
entropy. Given D(R), we can probably represent D(S | R) (denoting D(S)
relative to
dataset R) in small space, while still supporting the functionality of D
efficiently. Then,
given D(R) and D(S | R), we can either simulate D(S) directly, or decompress
it for
faster queries. A similar approach may also allow the construction of D(S) and
D(S | R)
efficiently, given D(R), R, and the differences between datasets S and R.
Our work clearly has links to persistent data structures and can be thought of
as a
special case where only the initial state and the final state are preserved,
the final
state being the net result of potentially many individual modifications, all
of which would
be represented by a persistent data structure.
\fi
\end{abstract}

%\newpage %for ALENEX format

\section{Introduction}

The suffix tree \cite{Weiner1973} is one of the most powerful bioinformatic tools to
answer complex queries on DNA and protein sequences \cite{Gus97,Ohl13,MBCT15}.
A serious problem that hampers its wider use on large genome sequences is its
size, which may be 10--20 bytes per character. In addition, the non-local
access patterns required by most interesting problems solved with suffix trees
complicate secondary-memory deployments. This problem has led to numerous 
efforts to reduce the suffix tree size by representing it as a set of {\em 
compact data structures} \cite{Sadakane2007,Fischer2009a,Ohlebusch2009,Ohlebusch2010,Russo2011,Gog2011a,Abeliuk2013,Navarro2014a}, leading to 
{\em compressed suffix trees (CSTs)}. Currently, the smallest 
CST is the so-called FCST \cite{Russo2011,Navarro2014a}, which uses about 
5 {\em bits} per character (bpc) but takes milliseconds to simulate suffix 
tree navigation operations. In the other extreme, Sadakane's CST 
\cite{Sadakane2007} takes about 12 bpc and operates in microseconds, and even 
nanoseconds for the simplest operations.

A space usage of 12 bpc may seem reasonable to handle, for example, one human
genome, which has about 3.1 billion bases: it can be operated within a
RAM of 4.5 GB (the representation contains the sequence as well). However, the
times where a single genome was analyzed are quickly giving way to databases
of genomes. Sequencing is becoming a routine activity where big companies 
produce thousands of genomes per day, and initiatives like the {\em 1000 Genomes
Project} \cite{Rozowsky2011} for challenging the current techniques with
multigenome-scale data are fluorishing.

Just storing 1000 human genomes using a 12 bpc CST requires almost 4.5 TB, which
is not a RAM size found in a commodity server. Even considering powerful 
commodity servers of 256 GB of RAM, we would need a cluster of 17 servers to 
handle such a collection of CSTs (compared to over 100 with classical suffix
tree implementations!). With the much smaller (and much slower) FCST, this would
drop to almost 7 servers. It is clear that further space reductions in the 
representation of CSTs would lead to reductions in hardware, communication, 
and energy costs when implementing complex searches over large genomic 
databases.

An important characteristic of those large genome databases is that they 
usually consist of the genomes of individuals of the same or closely related
species. This implies that the collections are highly {\em repetitive}, that
is, each genome can be obtained by concatenating a few pieces of other genomes
and adding a few new characters. When repetitiveness is considered, much higher
compression rates can be obtained in CSTs. For example, it is possible to reduce
the space to 1--2 bpc (albeit with operation times in the milliseconds)
\cite{Abeliuk2013}, or to 2--3 bpc with operation times in the microseconds
\cite{Navarro2014}. For example, using 2 bpc, our 1000 genomes could be handled
with just 3 servers of 256 GB of RAM. We note, however, that these CSTs index
the whole collection and not individual sequences, which makes a difference on
the types of queries that can be answered, and also make a distributed 
implementation less obviously scalable.

Compression algorithms best capture repetitiveness by using {\em grammar}
compression or {\em Lempel-Ziv} compression.\footnote{We refer to ``long-range''
repetitiveness, where similar texts may be found far away in the text
collection.} In the first case \cite{KY00,CLLPPSS05} one finds a context-free 
grammar that generates (only) the text collection, and the grammar is smaller 
as the collection is more repetitive. Rather than directly applying it on the 
text, the current CSTs for repetitive collections \cite{Abeliuk2013,Navarro2014}
apply grammar compression on the data structures that simulate the suffix tree.
Grammar compression yields relatively easy direct access to the compressed 
sequence \cite{BLRSRW15}. This makes it attractive compared to Lempel-Ziv 
compression \cite{ZL77}, despite the latter generally achieving less space.

Lempel-Ziv compression cuts the collection into {\em phrases}, each of which
has already appeared before. To extract the content of a phrase, one may have
to recursively extract the content at that earlier position, in a possibly long
chain of indirections.
Indeed, the only indices built on Lempel-Ziv compression \cite{KN13} or on
combinations of Lempel-Ziv and grammar compression \cite{GGKNP12,GGKNP14,GP15}
support only pattern matching, which is just one of the wide range of 
functionalities offered by suffix trees. The inability to access the data
at random positions lies at the heart of the research on indices built on
Lempel-Ziv compression. 

A simple way out of this limitation is the so-called {\em Relative Lempel-Ziv
(RLZ)} compression \cite{Kuruppu2010}, where one of the sequences is represented
in plain form and all the others can take phrases only from that reference
sequence. This enables immediate access for the symbols inside any copied
phrase (as no transitive referencing exists) and, at least in cases where a
good reference sequence is found, offers competitive compression compared with
classical Lempel-Ziv. In our case, taking any random genome per species as the
references is good enough; more sophisticated techniques have been studied
\cite{KPZ11}. Structures for direct access \cite{DG11,Ferrada2014}
and even for pattern matching \cite{DJSS14,Belazzougui2014} have been developed
on top of RLZ.

In this paper we show that RLZ is sufficiently friendly as a compression format
to build a CST on it, which uses less space as repetitiveness increases. 
On a collection of human genomes, we obtain as little as 3 bpc and operate 
within microseconds. This performance is comparable to that of a previous CST 
for this scenario \cite{Navarro2014}, but our CSTs have a different 
functionality: we have a separate CST for each sequence, instead of a single
CST for their concatenation. Depending on the application, one kind of CST or
the other is necessary. 

Our CST, called RCST, follows a trend of CSTs \cite{Fischer2009a,Ohlebusch2009,Ohlebusch2010,Gog2011a,Abeliuk2013} that use only a pattern-matching index 
(called suffix array) and an array with the length of the longest common prefix
between each suffix and the previous one in lexicographic order (called $\LCP$).
We use a suffix array tailored to RLZ compression \cite{Belazzougui2014} and 
also use RLZ to compress $\LCP$. On top of the $\LCP$ phrases we build a tree
of range minima that enables fast queries for range minimum queries, next and
previous smaller values, on $\LCP$ \cite{Abeliuk2013}. All the CST functionality
is built on those queries \cite{Fischer2009a}. Our main algorithmic contribution
is this RLZ-based representation of the $\LCP$ array with the required extra
functionality.

\iffalse
The main topic of the paper will be the relative CST, but we also have the RLZ
bitvector and the relative FM-index for read collections. The RLZ bitvector is
a nice idea that works well in practice, if we can just find applications for
it. Sorting suffixes in lexicographic order amplifies the differences between
the sequences, so it's probably going to be something unrelated to suffix
trees and suffix arrays.

Relative data compression is a well-established topic. Version control systems
store
revisions of files as insertions and deletions to earlier revisions. In
bioinformatics, individual
genomes are often represented by listing their differences to the reference
genome
of the same species. More generally, we can use relative Lempel-Ziv (RLZ)
parsing [10]
to represent a text as a concatenation of substrings of a related text.

Compressed data structures for repetitive data. Given similar datasets
S1,...,Sr, the
data structure D(S1,...,Sr) is often repetitive. If we compress these
repetitions, we can
represent and use the data structure in much smaller space. Compressed data
structures
achieve better compression than relative data structures, because they can
take advantage
of the redundancy between all datasets, instead of just between the current
dataset and
the reference dataset. The price is less flexibility, as the encoding of each
dataset may
depend on all the other datasets. While the construction of compressed data
structures for
multiple datasets requires dedicated algorithms and often also significant
computational
resources, we can easily distribute the construction of relative data
structures to multiple
systems, as well as add and remove datasets.

Persistent data structures preserve the state of the data structure before
each operation.
Relative data structures can be seen as a special case of persistent data
structures
that preserves only the initial and the final state, with more emphasis on
space-e"ciency.
Also, while research on persistent data structures concentrates on structures
that can be
dynamically updated, my emphasis is on static data structures that are smaller
and faster
to use.
\fi

\section{Background}

A \emph{string} $S[1,n] = s_{1} \dotso s_{n}$ is a sequence of
\emph{characters} over an \emph{alphabet} $\Sigma = \set{1, \dotsc, \sigma}$.
For indexing purposes, we often consider \emph{text} strings $T[1,n]$ that are
terminated by an \emph{endmarker} $T[n] = \$ = 0$ not occurring elsewhere in
the text. \emph{Binary} sequences are sequences over the alphabet $\set{0,1}$.
If $B[1,n]$ is a binary sequence, its \emph{complement} is binary sequence
$\complement{B}[1,n]$, with $\complement{B}[i] = 1 - B[i]$.

For any binary sequence $B[1,n]$, we define the \emph{subsequence} $S[B]$ of
string $S[1,n]$ as the concatenation of characters $s_{i}$ with $B[i] = 1$.
The complement $\complement{S}[B]$ of subsequence $S[B]$ is the subsequence
$S[\complement{B}]$. Contiguous subsequences $S[i,j]$ are called
\emph{substrings}. Substrings of the form $S[1,j]$ and $S[i,n]$, $i,j \in
[1,n]$, are called \emph{prefixes} and \emph{suffixes}, respectively. We
define the \emph{lexicographic order} among strings in the usual way.

\subsection{Full-text indexes}

The \emph{suffix tree} (\ST)~\cite{Weiner1973} of text $T$ is a trie
containing the suffixes of $T$, with unary paths compacted into single edges.
Because every internal nodes has degree at least two, there can be at most
$2n-1$ nodes, and the suffix tree can be stored in $\Oh(n \log n)$ bits. In
practice, this is at least $10n$ bytes for small texts~\cite{Kurtz1999}, and
more for large texts as the pointers grow larger. If $v$ is a node of a suffix
tree, we write $\pi(v)$ to denote the concatenation of the labels of the path
from the root to node $v$.

\emph{Suffix arrays} (\SA)~\cite{Manber1993} were introduced as a
space-efficient alternative to suffix trees. The suffix array $\mSA[1,n]$ of
text $T$ is an array of pointers to the suffixes of the text in lexicographic
order. In its basic form, the suffix array requires only $n \log n$ bits in
addition to the text, but its functionality is more limited than that of the
suffix tree. In addition to the suffix array, many algorithms also use the
\emph{inverse suffix array} $\mISA[1,n]$, with $\mSA[\mISA[i]] = i$ for all
$i$.

Let $\mlcp(S_{1}, S_{2})$ be the length of the \emph{longest common prefix}
(\LCP) of strings $S_{1}$ and $S_{2}$. The \LCP{}
\emph{array}~\cite{Manber1993} $\mLCP[1,n]$ of text $T$ stores the \LCP{}
lengths for lexicographically adjacent suffixes of $T$ as $\mLCP[i] =
\mlcp(T[\mSA[i-1],n], T[\mSA[i],n])$. Let $v$ be an internal node of the
suffix tree, $\ell = \abs{\pi(v)}$ the \emph{string depth} of node $v$, and
$\mSA[sp,ep]$ the corresponding suffix array interval. The following
properties hold for the \LCP{} \emph{interval} $\mLCP[sp,ep]$: i) $\mLCP[sp] <
\ell$; ii) $\mLCP[i] \ge \ell$ for all $sp < i \le ep$; iii) $\mLCP[i] = \ell$
for at least one $sp < i \le ep$; and iv) $\mLCP[ep+1] <
\ell$~\cite{Abouelhoda2004}.

Abouelhoda et al.~\cite{Abouelhoda2004} showed how traversals on the suffix
tree could be simulated using the suffix array, the \LCP{} array, and a
representation of the suffix tree topology based on \LCP{} intervals, paving
way for more space-efficient suffix tree representations.

\subsection{Compressed text indexes}

Data structures supporting \rank{} and \select{} queries over sequences are
the main building block of compressed text indexes. If $S$ is a sequence, we
define $\mrank_{c}(S,i)$ to be the number of occurrences of character $c$ in
prefix $S[1,i]$, while $\mselect_{c}(S,j)$ is the position of the occurrence
of rank $j$ in sequence $S$. A \emph{bitvector} is a representation of a
binary sequence $B$ supporting fast \rank{} and \select{} queries.
\emph{Wavelet trees} (\WT)~\cite{Grossi2003} use bitvectors to support \rank{}
and \select{} on general strings.

The \emph{Burrows-Wheeler transform} (\BWT)~\cite{Burrows1994} is a reversible
permutation $\mBWT[1,n]$ of text $T$. It is defined as $\mBWT[i] = T[\mSA[i] -
1]$ (with $\mBWT[i] = T[n]$, if $\SA[i] = 1$). Originally intended for data
compression, the Burrows-Wheeler transform has been widely used in
space-efficient text indexes, because it shares the combinatorial structure of
the suffix tree and the suffix array.

Let \LF{} be a function such that $\mSA[\mLF(i)] = \mSA[i] - 1$ (with
$\mSA[\mLF(i)] = n$, if $\mSA[i] = 1$). We can compute it as $\mLF(i) =
\mC[\mBWT[i]] + \mrank_{\mBWT[i]}(\mBWT, i)$, where $\mC[c]$ is the number of
occurrences of characters with lexicographical values smaller than $c$ in
\BWT. The inverse function of \LF{} is $\mPsi$, with $\mPsi(i) =
\mselect_{c}(\mBWT, i - \mC[c])$, where $c$ is the largest character value
with $\mC[c] < i$. With functions \LF{} and $\mPsi$, we can move forward and
backward in the text, while maintaining the lexicographic rank of the current
suffix. If sequence $S$ is not evident from the context, we write $\mLF_{S}$
and $\mPsi_{S}$.

\emph{Compressed suffix arrays} (\CSA) \cite{Ferragina2005a,Grossi2005} are
text indexes supporting similar functionality to the suffix array. This
includes the following queries: i) $\mfind(P) = [sp,ep]$ determines the
lexicographic range of suffixes starting with \emph{pattern} $P[1,\ell]$; ii)
$\mlocate(sp,ep) = \mSA[sp,ep]$ returns the starting positions of these
suffixes; and iii) $\mextract(i,j) = T[i,j]$ extracts substrings of the text.
In practice, the \find{} performance of compressed suffix arrays can be
competitive with suffix arrays, while \locate{} queries are orders of
magnitude slower~\cite{Ferragina2009a}. Typical index sizes are less than the
size of the uncompressed text.

The \emph{FM-index} (\FMI) \cite{Ferragina2005a} is a common type of
compressed suffix array. A typical implementation stores the \BWT{} in a
wavelet tree \cite{Grossi2003}. The index implements \find{} queries via a
process called \emph{backward search}. Let $[sp,ep]$ be the lexicographic
range of the suffixes of the text that start with suffix $P[i+1,\ell]$ of the
pattern. We can find the range matching suffix $P[i,\ell]$ with a
generalization of function \LF{} as
$$
\mLF([sp,ep],P[i]) =
[\mC[P[i]] + \mrank_{P[i]}(\mBWT, sp-1) + 1,
\mC[P[i]] + \mrank_{P[i]}(\mBWT, ep)].
$$

We support \locate{} queries by \emph{sampling} some suffix array pointers. If
we want to determine a value $\mSA[i]$ that has not been sampled, we can
compute it as $\mSA[i] = \mSA[j]+k$, where $\mSA[j]$ is a sampled pointer
found by iterating \LF{} $k$ times, starting from position $i$. Given
\emph{sample interval} $d$, the samples can be chosen in \emph{suffix order},
sampling $\mSA[i]$ at positions divisible by $d$, or in \emph{text order},
sampling $T[i]$ at positions divisible by $d$ and marking the sampled \SA{}
positions in a bitvector. Suffix-order sampling requires less space, often
resulting in better time/space trade-offs in practice, while text-order
sampling guarantees better worst-case performance. We also sample some \ISA{}
pointers for \extract{} queries. To extract $T[i,j]$, we find the nearest
sampled pointer after $\mISA[j]$, and traverse backwards to $T[i]$ with
function \LF.

\begin{table}
\centering{}
\caption{Typical compressed suffix tree operations.}\label{table:cst
operations}

\begin{tabular}{ll}
\hline
\noalign{\smallskip}
\textbf{Operation}  & \textbf{Description} \\
\noalign{\smallskip}
\hline
\noalign{\smallskip}
$\mRoot()$          & The root of the tree. \\
$\mLeaf(v)$         & Is node $v$ a leaf? \\
$\mAncestor(v,w)$   & Is node $v$ is an ancestor of node $w$? \\
\noalign{\smallskip}
$\mCount(v)$        & Number of leaves in the subtree with $v$ as the root. \\
$\mLocate(v)$       & Pointer to the suffix corresponding to leaf $v$. \\
\noalign{\smallskip}
$\mParent(v)$       & The parent of node $v$. \\
$\mFChild(v)$       & The first child of node $v$ in alphabetic order. \\
$\mNSibling(v)$     & The next sibling of node $v$ in alphabetic order. \\
$\mLCA(v,w)$        & The lowest common ancestor of nodes $v$ and $w$. \\
\noalign{\smallskip}
$\mSDepth(v)$       & \emph{String depth}: Length $\ell = \abs{\pi(v)}$ of the
label from the root to node $v$. \\
$\mTDepth(v)$       & \emph{Tree depth}: The depth of node $v$ in the suffix
tree. \\
$\mLAQ_{S}(v,d)$    & The highest ancestor of node $v$ with string depth at
least $d$. \\
$\mLAQ_{T}(v,d)$    & The ancestor of node $v$ with tree depth $d$. \\
\noalign{\smallskip}
$\mSLink(v)$        & Suffix link: Node $w$ such that $\pi(v) = c \pi(w)$ for
a character $c \in \Sigma$. \\
$\mSLink^{k}(v)$    & Suffix link iterated $k$ times. \\
\noalign{\smallskip}
$\mChild(v,c)$      & The child of node $v$ with edge label starting with
character $c$. \\
$\mLetter(v,i)$     & The character $\pi(v)[i]$. \\
\noalign{\smallskip}
\hline
\end{tabular}
\end{table}

\emph{Compressed suffix trees} (\CST) \cite{Sadakane2007} are compressed text
indexes supporting the full functionality of a suffix tree (see
Table~\ref{table:cst operations}). They combine a compressed suffix array, a
compressed representation of the \LCP{} array, and a compressed representation
of suffix tree topology. For the \LCP{} array, there are several common
representations:
\begin{itemize}
\item \LCPbyte{} \cite{Abouelhoda2004} stores the \LCP{} array as a byte
array. If $\mLCP[i] < 255$, the \LCP{} value is stored in the byte array.
Larger values are marked with a $255$ in the byte array and stored separately.
Because many texts produce small \LCP{} values, \LCPbyte{} usually requires
$n$ to $1.5n$ bytes of space.
\item We can store the \LCP{} array by using variable-length codes. \LCPdac{}
uses \emph{directly addressable codes} \cite{Brisaboa2009} for the purpose,
resulting in a structure that is typically somewhat smaller and somewhat
slower than \LCPbyte.
\item The \emph{permuted} \LCP{} (\PLCP) \emph{array} \cite{Sadakane2007}
$\mPLCP[1,n]$ is the \LCP{} array stored in text order and used as $\mLCP[i] =
\mPLCP[\mSA[i]]$. Because $\mPLCP[i+1] \ge \mPLCP[i]-1$, the array can be
stored as a bitvector of length $2n$ in $2n+\oh(n)$ bits. If the text is
repetitive, run-length encoding can be used to compress the bitvector to take
even less space \cite{Fischer2009a}. Because accessing \PLCP{} uses \locate,
it is much slower than the above two encodings.
\end{itemize}

Suffix tree topology representations are the main differences between the
various \CST{} proposals. While the compressed suffix arrays and \LCP{} arrays
are interchangeable, tree representation determines how various suffix tree
operations are implemented. There are three main families of compressed suffix
trees:
\begin{itemize}
\item \emph{Sadakane's compressed suffix tree} (\CSTsada) \cite{Sadakane2007}
uses a \emph{balanced parentheses} representation for the tree. Each node is
encoded as an opening parenthesis, followed by the encodings of its children
and a closing parenthesis. This can be encoded as a bitvector of length $2n'$,
where $n'$ is the number of nodes, requiring up to $4n+\oh(n)$ bits.
\CSTsada{} tends to be larger and faster than the other compressed suffix
trees \cite{Gog2011a,Abeliuk2013}.
\item The \emph{fully compressed suffix tree} (\FCST) of Russo et
al.~\cite{Russo2011,Navarro2014a} aims to use as little space as possible. It
does not require an \LCP{} array at all, and stores a balanced parentheses
representation for a sampled subset of suffix tree nodes in $\oh(n)$ bits.
Unsampled nodes are retrieved by following suffix links. \FCST{} is smaller
and much slower than the other compressed suffix trees
\cite{Russo2011,Abeliuk2013}.
\item Fischer et al.~\cite{Fischer2009a} proposed an intermediate
representation, \CSTnpr, based on \LCP{} intervals. Tree navigation is handled
by searching for the values defining the \LCP{} intervals. \emph{Range minimum
queries} $\mrmq(sp,ep)$ find the leftmost minimal value in $\mLCP[sp,ep]$,
while \emph{next/previous smaller value} queries $\mnsv(i)$/$\mpsv(i)$ find
the next/previous \LCP{} value smaller than $\mLCP[i]$. After the improvements
by various authors \cite{Ohlebusch2009,Ohlebusch2010,Gog2011a,Abeliuk2013},
the \CSTnpr{} is perhaps the most practical compressed suffix tree.
\end{itemize}

For typical texts and component choices, the size of compressed suffix trees
ranges from the $1.5n$ to $3n$ bytes of \CSTsada{} to the $0.5n$ to $n$ bytes
of \FCST{} \cite{Gog2011a,Abeliuk2013}. There are also some \CST{} variants
for repetitive texts, such as versioned document collections and collections
of individual genomes. Abeliuk et al.~\cite{Abeliuk2013} developed a variant
of \CSTnpr{} that can sometimes be smaller than $n$ bits, while achieving
similar performance as the \FCST. Navarro and Ordóñez \cite{Navarro2014} used
grammar-based compression for the tree representation of \CSTsada. The
resulting compressed suffix tree (\GCT) requires slightly more space than the
\CSTnpr{} of Abeliuk et al., while being closer to the non-repetitive
\CSTsada{} and \CSTnpr{} in performance.

\subsection{Relative Lempel-Ziv}\label{sect:rlz}

\emph{Relative Lempel-Ziv} (\RLZ) parsing \cite{Kuruppu2010} compresses
\emph{target} sequence $S$ relative to \emph{reference} sequence $R$. The
target sequence is represented as a concatenation of $z$ \emph{phrases} $w_{i}
= (p_{i}, \ell_{i}, c_{i})$, where $p_{i}$ is the starting position of the
phrase in the reference, $\ell_{i}$ is the length of the copied substring, and
$c_{i}$ is the \emph{mismatching} character. If phrase $w_{i}$ starts from
position $p'$ in the target, then $S[p',p'+\ell_{i}-1] =
R[p_{i},p_{i}+\ell_{i}-1]$ and $S[p'+\ell_{i}] = c_{i}$.

The shortest \RLZ{} parsing of the target sequence can be found in
(essentially) linear time. The algorithm builds a \CSA{} for the reverse of
the reference sequence, and then parses the target sequence greedily by using
backward searching. If the edit distance between the reference and the target
is $s$, we need at most $s$ phrases to represent the target sequence. On the
other hand, because the relative order of the phrases can be different in
sequences $R$ and $S$, the edit distance can be much larger than the number of
phrases in the shortest \RLZ{} parsing.

In a straightforward implementation, the \emph{phrase pointers} $p_{i}$ and
the mismatching characters $c_{i}$ can be stored in arrays $W_{p}$ and
$W_{c}$. These arrays take $z \log \abs{R}$ bits and $z \log \sigma$ bits,
respectively. To support random access in the target sequence, we can encode
phrase lengths as bitvector $W_{\ell}$ of length $\abs{S}$ \cite{Kuruppu2010}.
We set $W_{\ell}[j] = 1$, if $S[j]$ is the first character of a phrase. The
bitvector requires $z \log \frac{n}{z} + \Oh(z)$ bits, if we use the
\sdarray{} representation \cite{Okanohara2007}. To extract $S[j]$, we first
determine the phrase $w_{i}$, with $i = \mrank_{1}(W_{\ell}, j)$. If
$W_{\ell}[j+1] = 1$, we return the mismatching character $W_{c}[i]$. Otherwise
we determine the phrase offset with a \select{} query, and return character
$R[W_{p}[i] + j - \mselect_{1}(W_{\ell}, i)]$.

The \select{} query can be avoided by using \emph{relative pointers} instead
of absolute pointers \cite{Ferrada2014}. By setting $W_{p}[i] = p_{i} -
\mselect_{1}(W_{\ell}, i)$, the general case simplifies to $S[j] = R[W_{p}[i]
+ j]$. If most of the differences between the reference and the target
sequence are single-character \emph{substitutions}, $p_{i+1}$ will often be
$p_{i} + \ell_{i} + 1$. This corresponds to $W_{p}[i+1] = W_{p}[i]$ with
relative pointers, making \emph{run-length encoding} the pointer array
worthwhile.


\section{Relative FM-index}

The \emph{relative FM-index} (\RFM) \cite{Belazzougui2014} is a compressed
suffix array of a sequence relative to the \CSA{} of another sequence. We
write $\mRFM(S \mid R)$ to denote the relative FM-index of target sequence $S$
relative to reference sequence $R$. The index is based on approximating the
\emph{longest common subsequence} (\LCS) of $\mBWT(R)$ and $\mBWT(S)$, and
storing several structures based on the common subsequence. Given a
representation of $\mBWT(R)$ supporting \rank{} and \select{}, we can use the
relative index $\mRFM(S \mid R)$ to simulate \rank{} and \select{} on
$\mBWT(S)$.

\subsection{Basic index}

Assume that we have found a long common subsequence of sequences $X$ and $Y$.
We call positions $X[i]$ and $Y[j]$ \emph{lcs-positions}, if they are in the
common subsequence. If $B_{X}$ and $B_{Y}$ are the binary sequences marking
the common subsequence ($X[\select_1(B_{X},i)] = Y[\select_1(B_{Y},i)]$), we can map between the
corresponding lcs-positions in the two sequences with \rank{} and \select{}
operations. If $X[i]$ is an lcs-position, the corresponding position in
sequence $Y$ is $Y[\mselect_{1}(B_{Y}, \mrank_{1}(B_{X}, i))]$. We denote this
pair of \emph{lcs-bitvectors} $\mLCS(X,Y)$.

In its most basic form, the relative FM-index $\mRFM(S \mid R)$ only supports
\find{} queries by simulating \rank{} queries on $\mBWT(S)$. It does this by
storing $\mLCS(\BWT(R),\BWT(S))$ and the complements $\mCS(\mBWT(R))$ and
$\mCS(\mBWT(S))$ of the common subsequence. The lcs-bitvectors are compressed
using \emph{entropy-based compression} \cite{Raman2007}, while the complements
are stored in similar structures as the reference $\mBWT(R)$.

To compute $\mrank_{c}(\mBWT(S), i)$, we first determine the number of
lcs-positions in $\mBWT(S)$ up to position $S[i]$ with $k =
\mrank_{1}(B_{\mBWT(S)}, i)$. Then we find the lcs-position $k$ in $\mBWT(R)$
with $j = \mselect_{1}(B_{\mBWT(R)}, k)$. With these positions, we can compute
$$
\mrank_{c}(\mBWT(S), i) = \mrank_{c}(\mBWT(R), j) - \mrank_{c}(\mCS(\mBWT(R)),
j-k) + \mrank_{c}(\mCS(\mBWT(S)), i-k).
$$

\subsection{Relative select}

We can implement the entire functionality of a compressed suffix array with
\rank{} queries on the \BWT. However, if we use the \CSA{} in a compressed
suffix tree, we also need \select{} queries to support \emph{forward
searching} with $\mPsi$ and $\mChild$ queries. We can always implement
\select{} queries by binary searching with \rank{} queries, but the result
will be much slower than \rank{} queries.

A faster alternative to support \select{} queries in the relative FM-index
is to build a \emph{relative select} structure \rselect{}~\cite{Boucher2015}.
Let $\mF(X)$ be a sequence consisting of the characters of sequence $X$ in
sorted order. Alternatively, $\mF(X)$ is a sequence such that $\mF(X)[i] =
\mBWT(X)[\mPsi(i)]$. The relative select structure consists of bitvectors
$\mLCS(\mF(R), \mF(S))$, where $B_{\mF(R)}[i] = B_{\mBWT(R)}[\mPsi(i)]$ and
$B_{\mF(S)}[i] = B_{\mBWT(S)}[\mPsi(i)]$, as well as the \C{} array
$\mC(\mLCS)$ for the common subsequence.

To compute $\mselect_{c}(\mBWT(S), i)$, we first determine how many of
the first $i$ occurrences of character $c$ are lcs-positions with $k =
\mrank_{1}(B_{\mF(S)}, \mC(\mBWT(S))[c] + i) - \mC(\mLCS)[c]$. Then we check
from bit $B_{\mF(S)}[\mC(\mBWT(S))[c] + i]$ whether the occurrence we are
looking for is an lcs-position or not. If it is,
we find the position in $\mBWT(R)$ by computing $j = \mselect_{c}(\mBWT(R),
\mselect_{1}(B_{\mF(R)}, \mC(\mLCS)[c] + k))$, and then map $j$ to
$\mselect_{c}(\mBWT(S), i)$ by using $\mLCS(\mBWT(R), \mBWT(S))$. Otherwise we
find the occurrence in $\mCS(\mBWT(S))$ with $j = \mselect_{c}(\mCS(\mBWT(S)),
i-k)$, and return $\mselect_{c}(\mBWT(S), i) = \mselect_{0}(B_{\mBWT(S)}, j)$.

\subsection{Full functionality}

If we want the relative FM-index to support \locate{} and \extract{} queries,
we cannot build it from any common subsequence of $\mBWT(R)$ and $\mBWT(S)$.
We need a \emph{bwt-invariant subsequence} \cite{Belazzougui2014}, where the
relative order of the characters is the same in both the original sequences
and their Burrows-Wheeler transforms.

\begin{definition}\label{def:bwt-invariant}
Let $X$ be a common subsequence of $\mBWT(R)$ and $\mBWT(S)$, and let
$\mBWT(R)[i_{R}]$ and $\mBWT(S)[i_{S}]$ be the lcs-positions corresponding to
$X[i]$. Subsequence X is bwt-invariant if
$$
\mSA(R)[i_{R}] < \mSA(R)[j_{R}] \iff \mSA(S)[i_{S}] < \mSA(S)[j_{S}]
$$
for all positions $i, j \in \set{1, \dotsc, \abs{X}}$.
\end{definition}

In addition to the structures already mentioned, the full relative FM-index
has another pair of lcs-bitvectors, $\mLCS(R,S)$, which marks the
bwt-invariant subsequence in the original sequences. If $\mBWT(R)[i_{R}]$ and
$\mBWT(S)[i_{S}]$ are lcs-positions, we set $B_{R}[\mSA(R)[i_{R}]-1] = 1$ and
$B_{S}[\mSA(S)[i_{S}]-1] = 1$.\footnote{For simplicity, we assume that the
endmarker is not a part of the bwt-invariant subsequence. Hence $\mSA[i] > 1$
for all lcs-positions $\mBWT[i]$.} 

To compute the answer to a $\mlocate(i)$ query, we start by iterating
$\mBWT(S)$ backwards with \LF{} queries, until we find an lcs-position
$\mBWT(S)[i']$ after $k$ steps. Then we map position $i'$ to the corresponding
position $j'$ in $\mBWT(R)$ by using $\mLCS(\mBWT(R),\mBWT(S))$. Finally we
determine $\mSA(R)[j']$ with a \locate{} query in the reference index, and map
the result to $\mSA(S)[i']$ by using $\mLCS(R,S)$.\footnote{If $\mBWT(S)[i']$
and $\mBWT(R)[j']$ are lcs-positions, the corresponding lcs-positions in the
original sequences are $S[\mSA(S)[i']-1]$ and $R[\mSA(R)[j']-1]$.} The result
of the $\mlocate(i)$ query is $\mSA(S)[i']+k$.

The $\mISA(S)[i]$ access required for \extract{} queries is supported in a
similar way. We find the lcs-position $S[i+k]$ for the smallest $k \ge 0$, and
map it to the corresponding position $R[j]$ by using $\mLCS(R,S)$. Then we
determine $\mISA(R)[j+1]$ by using the reference index, and map it back to
$\mISA(S)[i+k+1]$ with $\mLCS(\mBWT(R),\mBWT(S))$. Finally we iterate
$\mBWT(S)$ backwards $k+1$ steps with \LF{} queries to find $\mISA(S)[i]$.

If the target sequence contains long
\emph{insertions} not present in the reference, we may also want to include
some \SA{} and \ISA{} samples for querying those regions.

\subsection{Finding bwt-invariant subsequence}

With the basic relative FM-index, we approximate the longest common
subsequence of $\mBWT(R)$ and $\mBWT(S)$ by partitioning the \BWT{}s according
to lexicographic contexts, finding the longest common subsequence for each
pair of substrings in the partitioning, and concatenating the results. The
algorithm is fast and easy to parallelize, and quite space-efficient. As such,
\RFM{} construction is practical, having been tested with datasets of hundreds
of gigabytes in size.

To find a bwt-invariant subsequence, we start by \emph{matching} each suffix
of the reference sequence with the lexicographically nearest suffixes of the
target sequence. Unlike in the original algorithm \cite{Belazzougui2014}, we
only match suffixes that are lexicographically adjacent in the \emph{mutual
suffix array} of the two sequences.

\begin{definition}
Let $R$ and $S$ be two sequences, and let $\mSA = \mSA(RS)$ and $\mISA =
\mISA(RS)$. The \emph{left match} of suffix $R[i,\abs{R}]$ is the suffix
$S[\mSA[\mISA[i]-1] - \abs{R}, \abs{S}]$, if $\mISA[i] > 1$ and
$\mSA[\mISA[i]-1]$ points to a suffix of $S$ ($\mSA[\mISA[i]-1] > \abs{R}$).
The \emph{right match} of suffix $R[i,\abs{R}]$ is the suffix
$S[\mSA[\mISA[i]+1] - \abs{R}, \abs{S}]$, if $\mISA[i] < \abs{RS}$ and
$\mSA[\mISA[i]+1]$ points to a suffix of $S$.
\end{definition}

Instead of using the mutual suffix array, we can use $\mCSA(R)$, $\mCSA(S)$,
and the \emph{merging bitvector} $B_{R,S}$ of length $\abs{RS}$. We set
$B_{R,S}[i] = 1$, if $\mSA(RS)[i]$ points to a suffix of $S$. We can build the
merging bitvector in $\Oh(\abs{S} \cdot t_{\mLF})$ time, where $t_{\mLF}$ is
the time required for an \LF{} query, by extracting $S$ from $\mCSA(S)$ and
backward searching for it in $\mCSA(R)$ \cite{Siren2009}. Suffix
$R[i,\abs{R}]$ has a left (right) match, if $B_{R,S}[\mselect_{0}(B_{R,S},
\mISA(R)[i])-1] = 1$ ($B_{R,S}[\mselect_{0}(B_{R,S}, \mISA(R)[i])+1] = 1)$).

Our next step is building the \emph{match arrays} $\mleft$ and $\mright$,
which correspond to the arrays $A[\cdot][2]$ and $A[\cdot][1]$ in the original
algorithm. This is done by traversing $\mCSA(R)$ backwards from
$\mISA(R)[\abs{R}] = 1$ with \LF{} queries and following the left and the
right matches of the current suffix. During the traversal, we maintain
invariant $j = \mSA(R)[i]$ with $(i,j) \leftarrow (\mLF_{R}(i), j-1)$. If
suffix $R[i,\abs{R}]$ has a left (right) match, we use shorthand $l(i) =
\mrank_{1}(B_{R,S}, \mselect_{0}(B_{R,S}, i)-1)$ ($r(i) = \mrank_{1}(B_{R,S},
\mselect_{0}(B_{R,S}, i)+1)$) to refer to its position in $\mCSA(S)$.

We say that suffixes $R[i,\abs{R}]$ and $R[i+1,\abs{R}]$ have the same left
match, if $l(i) = \mLF_{S}(l(i+1))$. Let $R[i,\abs{R}]$ to $R[i+\ell,\abs{R}]$
be a maximal run of suffixes having the same left match, with suffixes
$R[i,\abs{R}]$ to $R[i+\ell-1,\abs{R}]$ starting with the same characters as
their left matches.\footnote{The first character of a suffix can be determined
by using the $\mC$ array.} We find the left match of suffix $R[i,\abs{R}]$ as
$i' = \mSA(S)[l(i)]$ by using $\mCSA(S)$, and set $\mleft[i,i+\ell-1] =
[i',i'+\ell-1]$. The right match array $\mright$ is built in a similar way.

The match arrays require $2\abs{R} \log \abs{S}$ bits of space. If sequences
$R$ and $S$ are similar, the runs in the arrays tend to be long. Hence we can
run-length encode the match arrays to save space. The traversal takes
$\Oh(\abs{R} \cdot (t_{\mLF} + t_{\mrank} + t_{\mselect}) + rd \cdot
t_{\mLF})$ time, where $t_{\mrank}$ and $t_{\mselect}$ denote the time
required by \rank{} and \select{} operations, $r$ is the number of runs in the
two arrays, and $d$ is the suffix array sample interval in
$\mCSA(S)$.\footnote{The time bound assumes text order sampling.}

The final step is finding the longest increasing subsequence $X$ of arrays
$\mleft$ and $\mright$, which corresponds to a common subsequence of $R$ and
$S$. More precisely, we want to find a binary sequence $B_{R}[1,\abs{R}]$,
which marks the common subsequence in $R$, and an integer sequence $X$, which
contains the positions of the common subsequence in $S$. The goal is to make
sequence $X$ strictly increasing and as long as possible, with
$X[\mrank_{1}(B_{R}, i)]$ being either $\mleft[i]$ or $\mright[i]$. This can
be done in $\Oh(\abs{R} \log \abs{R})$ time with $\Oh(\abs{R} \log \abs{R})$
bits of additional working space with a straightforward modification of the
dynamic programming algorithm for finding the longest increasing subsequence.
While the dynamic programming tables can be run-length encoded, the time and
space savings are negligible or even non-existent in practice.

As sequence $X$ is strictly increasing, we can convert it into binary sequence
$B_{S}[1,\abs{S}]$, marking the values in sequence $X$ with \onebit{}s.
Afterwards, we can consider binary sequences $B_{R}$ and $B_{S}$ as the
lcs-bitvectors $\mLCS(R,S)$. Because every suffix of $R$ starts with the same
character as its matches stored in the $\mleft$ and $\mright$ arrays,
subsequences $R[B_{R}]$ and $S[B_{S}]$ are identical. As each suffix
$R[i,\abs{R}]$ with $B_{R}[i] = 1$ is paired with its left match or right
match in sequence $S$, no other suffix of $R$ or $S$ is lexicographically
between the two paired suffixes.

For any $i$, let $i_{R} = \mselect_{1}(B_{R}, i)$ and $i_{S} =
\mselect_{1}(B_{S}, i)$ be the lcs-positions of rank $i$. Then,
$$
\mISA(R)[i_{R}] < \mISA(R)[j_{R}] \iff \mISA(S)[i_{S}] < \mISA(S)[j_{S}]
$$
for any $i,j \le \abs{X}$, which is equivalent to the condition in
Definition~\ref{def:bwt-invariant}. We can convert $\mLCS(R,S)$ to
$\mLCS(\mBWT(R),\mBWT(S))$ in $\Oh((\abs{R}+\abs{S}) \cdot t_{\mLF})$ time by
traversing $\mCSA(R)$ and $\mCSA(S)$ backwards. The resulting subsequence of
$\mBWT(R)$ and $\mBWT(S)$ is bwt-invariant.

Note that the full relative FM-index is more limited than the basic index,
because it does not handle \emph{substring moves} very well. Let $R = xy$ and
$S = yx$, for two random sequences $x$ and $y$ of length $n/2$ each. Because
$\mBWT(R)$ and $\mBWT(S)$ are very similar, we can expect to find a common
subsequence of length almost $n$. On the other hand, the length of the longest
bwt-invariant subsequence is around $n/2$, because we can either pair the
suffixes of $x$ or the suffixes of $y$ in $R$ and $S$, but not both.


\section{Relative compressed suffix tree}

The \emph{relative compressed suffix tree} (\RCST) is a \CSTnpr{} of the
target sequence relative to a \CST{} of the reference sequence. It consists of
two major components: the relative FM-index with full functionality and the
\emph{relative} \LCP{} (\RLCP) \emph{array}. The optional relative select
structure can be generated or loaded from disk to speed up algorithms based on
forward searching. The \RLCP{} array is based on \RLZ{} parsing, while the
support for \nsv/\psv/\rmq{} queries is based on a minima tree over the
phrases.

\subsection{Relative \LCP{} array}

Given \LCP{} array $\mLCP[1,n]$, we define the \emph{differential} \LCP{}
\emph{array} $\mDLCP[1,n]$ as $\mDLCP[1] = \mLCP[1]$ and $\mDLCP[i] = \mLCP[i]
- \mLCP[i-1]$ for $i > 1$. If $\mBWT[i,j] = c^{j+1-i}$ for some $c \in
\Sigma$, then $\mLCP[\mLF(i)+1,\mLF(j)]$ is the same as $\mLCP[i+1,j]$, with
each value incremented by $1$ \cite{Fischer2009a}. This means
$\mDLCP[\mLF(i)+2,\mLF(j)] = \mDLCP[i+2,j]$, making the \DLCP{} array of a
repetitive text compressible with grammar-based compression
\cite{Abeliuk2013}.

We make a similar observation in the relative setting. If target sequence $S$
is similar to the reference sequence $T$, then their \LCP{} arrays should also
be similar. If there are long identical ranges $\mLCP(R)[i,i+k] =
\mLCP(S)[j,j+k]$, the corresponding \DLCP{} ranges $\mDLCP(R)[i+1,i+k]$ and
$\mDLCP(S)[j+1,j+k]$ are also identical. Hence we can use \RLZ{} parsing to
compress either the original \LCP{} array or the \DLCP{} array.

While the identical ranges are a bit longer in the \LCP{} array, we opt to
compress the \DLCP{} array, because it behaves better when there are long
repetitions in the sequences. In particular, assembled genomes often have long
runs of character $N$, which correspond to regions of very large \LCP{}
values. If the runs are longer in the target sequence than in the reference
sequence, the \RLZ{} parsing of the \LCP{} array will have many phrases
containing only the mismatching character. The corresponding ranges in the
\DLCP{} array typically consist of values $\set{-1, 0, 1}$, making them much
easier to compress.

We create an \RLZ{} parsing of $\mDLCP(S)$ relative to $\mDLCP(R)$, while
using $\mLCP(R)$ as the reference afterwards. The reference is stored in a
structure we call \slarray, which is very similar to \LCPbyte.
\cite{Abouelhoda2004}. Small values $\mLCP(R)[i] < 255$ are stored in a byte
array, while large values $\mLCP(R)[i] \ge 255$ are marked with a $255$ in the
byte array and stored separately. To quickly find the large values, we also
build a $\mrank_{255}$ structure over the byte array. The \slarray{} provides
reasonably fast random access and very fast sequential access to the
underlying array.

The \RLZ{} parsing produces a sequence of phrases $w_{i} = (p_{i}, \ell_{i},
c_{i})$ (see Section~\ref{sect:rlz}). Because some queries involve
decompressing an entire phrase, we limit the maximum phrase length to $1024$.
We use absolute phrase pointers, as the run-length encoding of relative
pointers does not work too well with the \DLCP{} array, and because we need
access to the beginning of the phrase anyway. Phrase lengths are encoded in
the $W_{\ell}$ bitvector in the usual way. We convert the mismatching \DLCP{}
values $c_{i}$ into absolute \LCP{} values in the mismatch array $W_{c}$, and
store them as an \slarray. The mismatching values are used as \emph{absolute
samples} for the differential encoding.

To access $\mLCP(S)[j]$, we determine phrase $w_{i}$ as usual, and check
whether we should return the mismatch $W_{c}[j]$. If not, we determine the
starting position $s_{i} = \mselect_{1}(W_{\ell}, i)$ of the phrase in the
reference. Now we can compute the solution as
\begin{align*}
\mLCP(S)[j] &= W_{c}[i-1] + \sum_{k = W_{p}[i]}^{W_{p}[i]+j-s_{i}}
(\mLCP(R)[k] - \mLCP(R)[k-1]) \\
&= W_{c}[i-1] + \mLCP(R)[W_{p}[i]+j-s_{i}] - \mLCP(R)[W_{p}[i]-1],
\end{align*}
where $W_{c}[0] = \mLCP(R)[0] = 0$. After finding $\mLCP(S)[j]$, accessing
$\mLCP(S)[j-1]$ and $\mLCP(S)[j+1]$ is fast, as long as we do not cross phrase
boundaries.

\subsection{Supporting \nsv/\psv/\rmq{} queries}

Suffix tree topology can be inferred from the \LCP{} array with range minimum
queries (\rmq) and next/previous smaller value (\nsv/\psv) queries
\cite{Fischer2009a}. Some tree operations can be implemented more efficiently
if we also support \emph{next/previous smaller or equal value} (\nsev/\psev)
queries \cite{Abeliuk2013}. Query $\mnsev(i)$ ($\mpsev(i)$) finds the next
(previous) value smaller than or equal to $\mLCP[i]$.

In order to support the queries, we build a $4$-ary \emph{minima tree} over
the phrases of the \RLZ{} parsing. Each leaf node stores the smallest \LCP{}
value in the corresponding phrase, while each internal node stores the
smallest value in the subtree. Internal nodes are created and stored in a
levelwise fashion, so that each internal node, except perhaps the rightmost
one of its level, has $4$ children.

We encode the minima tree as two arrays. The smallest \LCP{} values are
stored in $M_{\mLCP}$, which we encode as an \slarray. Plain array $M_{L}$
stores the starting offset of each level in $M_{\mLCP}$, with the leaves
stored starting from offset $M_{L}[1] = 1$. If $i$ is a minima tree node
located at level $j$, the corresponding minimum value is $M_{\mLCP}[i]$, the
parent of the node is $M_{L}[j+1] + (i - M_{L}[j]) / 4$, and its first child
is $M_{L}[j-1] + 4 \cdot (i - M_{L}[j])$.

A range minimum query $\mrmq(sp,ep)$ starts by finding the minimal range of
phrases $w_{l}, \dotsc, w_{r}$ covering the query and the maximal range of
phrases $w_{l'}, \dotsc, w_{r'}$ contained in the query (note $l \le l' \le
l+1$ and $r-1 \le r' \le r$). Then we use the
minima tree to find the leftmost minimum value $j = M_{\mLCP}[k]$ in
$M_{\mLCP}[l',r']$, and find the leftmost occurrence $\mLCP[i] = j$ in phrase
$w_{k}$. If $l < l'$ and $M_{\mLCP}[l] \le j$, we decompress phrase $w_{l}$
and find the leftmost minimum value $\mLCP[i'] = j'$ (with $i' \ge sp$) in the
phrase. If $j' \le j$, we update $(i,j) \leftarrow (i',j')$. Finally we check
phrase $w_{r}$ in a similar way, if $r > r'$ and $M_{\mLCP}[r] < j$. The answer
to the range minimum query is $\mLCP[i] = j$, so we return
$(i,j)$.\footnote{The definition of the query only calls for the leftmost
minimum position $i$. We also return $\mLCP[i] = j$, because suffix tree
operations often need it.} Finally, the particular case where no phrase is
contained in $[sp,ep]$ is handled by sequentially scanning one or two phrases
in $\LCP$.

The remaining queries are all similar to each other. In order to answer query
$\mnsv(i)$, we start by finding the phrase $w_{k}$ containing position $i$,
and then determining $\mLCP[i]$. Next we scan the rest of the phrase to see
whether there is a smaller value $\mLCP[j] < \mLCP[i]$ later in the phrase. If
so, we return $(j,\mLCP[j])$. Otherwise we traverse the minima tree to find
the smallest $k' > k$ with $M_{\mLCP}[k'] < \mLCP[i]$. Finally we decompress
phrase $w_{k'}$, find the leftmost position $j$ with $\mLCP[j] < \mLCP[i]$,
and return $(j,\mLCP[j])$.


\section{Experiments}

We have implemented the relative compressed suffix tree in C++, extending the
old relative FM-index implementation.\footnote{The implementation is available
at \url{https://github.com/jltsiren/relative-fm}.} The implementation is based
on the \emph{Succinct Data Structure Library (SDSL) 2.0}~\cite{Gog2014b}. Some
parts of the implementation have been parallelized using \emph{OpenMP} and the
\emph{libstdc++ parallel mode}.

We used the SDSL implementation of the \emph{succinct suffix array} (\SSA{})
\cite{Ferragina2007a,Maekinen2005} as our reference \CSA{} and our baseline
index. \SSA{} encodes the Burrows-Wheeler transform as a \emph{Huffman-shaped
wavelet tree}, combining very fast queries with size close to the
\emph{order\nobreakdash-$0$ empirical entropy}. These properties make it the
index of choice for DNA sequences \cite{Ferragina2009a}. Due to the long runs
of character $N$, \emph{fixed block compression boosting}
\cite{Kaerkkaeinen2011} could reduce index size without significantly
increasing query times. Unfortunately there is no implementation capable of
handling multi-gigabyte datasets available.

We sampled \SA{} in suffix order and \ISA{} in text order. In \SSA, the sample
intervals were $17$ for \SA{} and $64$ for \ISA. In \RFM, we used sample
interval $257$ for \SA{} and $512$ for \ISA{} to handle the regions that do
not exist in the reference. The sample intervals for suffix order sampling
were primes due to the long runs of character $N$ in the assembled genomes. If
the number of long runs of character $N$ in the indexed sequence is even, the
lexicographic ranks of almost all suffixes in half of the runs are odd, and
those runs are almost completely unsampled. This can be avoided by making the
sample interval and the number of runs \emph{relatively prime}.

The experiments were run on a computer cluster running LSF 9.1.1.1 on Ubuntu
12.04 with Linux kernel 2.6.32. For most experiments, we used cluster nodes
with two 16-core AMD Opteron 6378 processors and 256 gigabytes of memory. Some
index construction jobs may have run on nodes with two 12-core AMD Opteron
6174 processors and 80 or 128 gigabytes of memory. All query experiments were
run single-threaded with no other jobs in the same node. Index construction
used 8 parallel threads, but there may have been other jobs running on the
same nodes at the same time.

As our primary target sequence, we used the \emph{maternal haplotypes} of the
\emph{1000 Genomes Project individual NA12878}~\cite{Rozowsky2011}. As the
target sequence, we used the 1000 Genomes Project version of the \emph{GRCh37
assembly} of the \emph{human reference
genome}.\footnote{\url{ftp://ftp.1000genomes.ebi.ac.uk/vol1/ftp/technical/reference/}}
Because NA12878 is female, we also created a reference sequence without the
chromosome~Y.

In the following, a basic FM-index is an index supporting only \find{}
queries, while a full index also supports \locate{} and \extract{} queries.

\subsection{Indexes and their sizes}

Table~\ref{table:construction} lists the resource requirements for building
the relative indexes, assuming that we have already built the corresponding
non-relative structures for the sequences. As a comparison, building an
FM-index for a human genome typically takes 16--17 minutes and 25--26
gigabytes of memory. While the construction of the basic \RFM{} index is
highly optimized, the other construction algorithms are just the first
implementations.

The construction times for the relative \CST{} do not include the time
required for indexing the \DLCP{} array of the reference sequence. While this
takes another two hours, it only needs to be done once for every reference
sequence. Building the optional \rselect{} structures takes 9--10 minutes and
around $\abs{R} + \abs{S}$ bits of working space in addition to \RFM{} and
\rselect.

\begin{table}
\caption{Sequence lengths and resources used by index construction for NA12878
relative to the human reference genome with and without chromosome~Y. Approx
and Inv denote the approximate \LCS{} and the bwt-invariant subsequence.
Sequence lengths are in millions of base pairs, while construction resources
are in minutes of wall clock time and gigabytes of
memory.}\label{table:construction}
\setlength{\extrarowheight}{2pt}
\setlength{\tabcolsep}{3pt}
\begin{center}
\begin{tabular}{c|cccc|cc|cc|cc}
\hline
 &
\multicolumn{4}{c|}{\textbf{Sequence length}} &
\multicolumn{2}{c|}{\textbf{\RFM{} (basic)}} &
\multicolumn{2}{c|}{\textbf{\RFM{} (full)}} &
\multicolumn{2}{c}{\textbf{\RCST}} \\
\textbf{ChrY} &
\textbf{Reference} & \textbf{Target} & \textbf{Approx} & \textbf{Inv} &
\textbf{Time} & \textbf{Space} &
\textbf{Time} & \textbf{Space} &
\textbf{Time} & \textbf{Space} \\
\hline
yes & 3096M & 3036M & 2992M & 2980M & 2.35 min & 4.96 GB & 238 min & 83.7 GB &
379 min & 99.0 GB \\
no  & 3036M & 3036M & 2991M & 2980M & 2.28 min & 4.86 GB & 214 min & 82.3 GB &
398 min & 97.2 GB \\
\hline
\end{tabular}
\end{center}
\end{table}

The sizes of the final indexes are listed in Table~\ref{table:indexes}. While
a basic \RFM{} index is 5\nobreakdash--6 times smaller than a basic \SSA, the
full \RFM{} is 4.4\nobreakdash--5 times smaller than the full \SSA. The
\RLCP{} array is about twice as large as the full \RFM{} index, increasing the
total size of the \RCST{} to 3.2\nobreakdash--4.3 bits per character. The
optional relative select structure is almost as large as the basic \RFM{}
index. As the relative structures are significantly larger relative to a male
reference than relative to a female reference, keeping a separate female
reference seems worthwhile, if there are more than a few female genomes among
the target sequences.

\begin{table}
\caption{Various indexes for NA12878 relative to the human reference genome
with and without chromosome~Y. Index sizes are in megabytes and in bits per
character.}\label{table:indexes}
\setlength{\extrarowheight}{2pt}
\setlength{\tabcolsep}{3pt}
\begin{center}
\begin{tabular}{c|cc|cc|cccc}
\hline
 &
\multicolumn{2}{c|}{\textbf{\SSA}} &
\multicolumn{2}{c|}{\textbf{\RFM}} &
\multicolumn{4}{c}{\textbf{\RCST}} \\
\textbf{ChrY} &
\textbf{Basic} & \textbf{Full} &
\textbf{Basic} & \textbf{Full} &
\textbf{\RFM} & \textbf{\RLCP} & \textbf{Total} & \textbf{\rselect} \\
\hline
\multirow{2}{*}{yes} &  1090 MB &  1953 MB &   218 MB &   447 MB &   447 MB &
1100 MB &  1547 MB &   190 MB \\
                     & 3.01 bpc & 5.42 bpc & 0.60 bpc & 1.23 bpc & 1.23 bpc &
3.04 bpc & 4.27 bpc & 0.52 bpc \\
\hline
\multirow{2}{*}{no}  &  1090 MB &  1953 MB &   181 MB &   395 MB &   395 MB &
750 MB &  1145 MB &   163 MB \\
                     & 3.01 bpc & 5.42 bpc & 0.50 bpc & 1.09 bpc & 1.09 bpc &
2.07 bpc & 3.16 bpc & 0.45 bpc \\
\hline
\end{tabular}
\end{center}
\end{table}

Tables~\ref{table:rfm components} and~\ref{table:rlcp components} list
the sizes of the individual components of the relative FM-index and the
\RLCP{} array. Including the chromosome~Y in the reference increases the sizes
of almost all relative components, with the exception of $\mCS(\mBWT(S))$ and
$\mLCS(R,S)$. In the first case, the common subsequence still covers
approximately the same positions in $\mBWT(S)$ as before. In the second case,
chromosome~Y appears in bitvector $B_{R}$ as a long run of \zerobit{}s, which
compresses well. The components of a full \RFM{} index are slightly larger
than the corresponding components of a basic \RFM{} index, because the
bwt-invariant subsequence is slightly shorter than the approximate longest
common subsequence (see Table~\ref{table:construction}).

\begin{table}
\caption{Breakdown of component sizes in the \RFM{} index for NA12878 relative
to the human reference genome with and without chromosome~Y in bits per
character.}\label{table:rfm components}
\setlength{\extrarowheight}{2pt}
\setlength{\tabcolsep}{3pt}
\begin{center}
\begin{tabular}{c|cc|cc}
\hline
 & \multicolumn{2}{c|}{\textbf{Basic \RFM}} & \multicolumn{2}{c}{\textbf{Full
\RFM}} \\
\textbf{ChrY} & \textbf{yes} & \textbf{no} & \textbf{yes} & \textbf{no} \\
\hline
\textbf{\RFM}              & \textbf{0.60 bpc} & \textbf{0.50 bpc} &
\textbf{1.23 bpc} & \textbf{1.09 bpc} \\
$\mCS(\mBWT(R))$           &          0.11 bpc &          0.04 bpc &
0.12 bpc &          0.05 bpc \\
$\mCS(\mBWT(S))$           &          0.04 bpc &          0.04 bpc &
0.05 bpc &          0.05 bpc \\
$\mLCS(\mBWT(R),\mBWT(S))$ &          0.45 bpc &          0.42 bpc &
0.52 bpc &          0.45 bpc \\
$\mLCS(R,S)$               &                -- &                -- &
0.35 bpc &          0.35 bpc \\
\SA{} samples              &                -- &                -- &
0.12 bpc &          0.12 bpc \\
\ISA{} samples             &                -- &                -- &
0.06 bpc &          0.06 bpc \\
\hline
\end{tabular}
\end{center}
\end{table}

\begin{table}
\caption{Breakdown of component sizes in the \RLCP{} array for NA12878
relative to the human reference genome with and without chromosome~Y in bits
per character. The components are phrase pointers, phrase boundaries in the
target \LCP{} array, mismatching characters, and the minima
tree.}\label{table:rlcp components}
\setlength{\extrarowheight}{2pt}
\setlength{\tabcolsep}{3pt}
\begin{center}
\begin{tabular}{c|cccc|c}
\hline
\textbf{ChrY} & $\mathbf{W_{p}}$  & $\mathbf{W_{\ell}}$ & $\mathbf{W_{c}}$ &
$\mathbf{M_{\mLCP}}$ & \textbf{Total} \\
\hline
yes & 1.44 bpc & 0.34 bpc & 0.54 bpc & 0.71 bpc & 3.04 bpc \\
no  & 1.11 bpc & 0.27 bpc & 0.30 bpc & 0.39 bpc & 2.07 bpc \\
\hline
\end{tabular}
\end{center}
\end{table}

\subsection{Query times}

Average query times for the basic queries can be seen in Tables~\ref{table:rfm
queries} and~\ref{table:rlcp queries}. For \LF{} and \Psiop{} queries
with the full \SSA{}, the full \RFM, and the full \RFM{} augmented with
\rselect, the results are similar to the earlier ones with basic indexes
\cite{Boucher2015}. Random access to the \RLCP{} array is about 20 times
slower than to the \LCP{} array. The \LCP{} array provides sequential access
iterators, which are much faster than using random access sequentially. The
\RLCP{} array does not have such iterators, because subsequent phrases are
often copied from different parts of the reference. However, queries based on
the minima tree (\nsv, \psv, and \rmq) use fast sequential access to the
\RLCP{} array inside a phrase. \SSA{} and the \LCP{} array are consistently
slower when the reference does not contain chromosome~Y, even though the
structures are identical in either case. This is probably a memory management
artifact that depends on other memory allocations.

\begin{table}
\caption{Query times with \SSA{} and the \RFM{} index for NA12878 relative to
the human reference genome with and without chromosome~Y in microseconds. The
query times are averages over 10~million random queries.}\label{table:rfm
queries}
\setlength{\extrarowheight}{2pt}
\setlength{\tabcolsep}{3pt}
\begin{center}
\begin{tabular}{c|cc|cc|c}
\hline
 & \multicolumn{2}{c|}{\textbf{\SSA}} & \multicolumn{2}{c|}{\textbf{\RFM}} &
\textbf{\rselect} \\
\textbf{ChrY} & \textbf{\LF} & \textbf{\Psiop} & \textbf{\LF} &
\textbf{\Psiop} & \textbf{\Psiop} \\
\hline
yes & 0.560 \mus & 1.139 \mus & 3.980 \mus & 47.249 \mus & 6.277 \mus \\
no  & 0.627 \mus & 1.563 \mus & 3.861 \mus & 55.068 \mus & 6.486 \mus \\
\hline
\end{tabular}
\end{center}
\end{table}

\begin{table}
\caption{Query times with the reference \LCP{} array and the \RLCP{} array for
NA12878 relative to the human reference genome with and without chromosome~Y
in microseconds. For the random queries, the query times are averages over
100~million queries. The range lengths for \rmq{} queries were $16^{k}$ (for
$k \ge 1$) with probability $0.5^{k}$. For sequential access, the times are
averages per position for scanning the entire \LCP{} array.}\label{table:rlcp
queries}
\setlength{\extrarowheight}{2pt}
\setlength{\tabcolsep}{3pt}
\begin{center}
\begin{tabular}{c|cc|ccccc}
\hline
 & \multicolumn{2}{c|}{\textbf{\LCP{} array}} &
\multicolumn{5}{c}{\textbf{\RLCP{} array}} \\
\textbf{ChrY} & \textbf{Random} & \textbf{Sequential} & \textbf{Random} &
\textbf{Sequential} & \textbf{\nsv} & \textbf{\psv} & \textbf{\rmq} \\
\hline
yes & 0.052 \mus & 0.001 \mus & 1.096 \mus & 0.119 \mus & 1.910 \mus & 1.935
\mus & 2.769 \mus \\
no  & 0.070 \mus & 0.001 \mus & 1.263 \mus & 0.124 \mus & 1.801 \mus & 1.923
\mus & 2.605 \mus \\
\hline
\end{tabular}
\end{center}
\end{table}

We also tested the \locate{} performance of the full \RFM{} index, and
compared it to \SSA. We built \SSA{} with \SA{} sample intervals $7$, $17$,
$31$, $61$, and $127$ for the reference and the target sequence, using only
the reference without chromosome~Y. \ISA{} sample interval was set to the
maximum of $64$ and the \SA{} sample interval. We then built \RFM{} for the
target sequence, and extracted 2~million random patterns of length $32$,
consisting of characters $ACGT$, from the target sequence. The time/space
trade-offs for \locate{} queries with these patterns can be seen in
Figure~\ref{fig:locate}. While the \RFM{} index was 8.6x slower than \SSA{}
with sample interval $7$, the absolute performance difference remained almost
constant with longer sample intervals. With sample interval $127$, \RFM{} was
only 1.2x slower than \SSA.

\begin{figure}
\begin{center}
\includegraphics{locate.pdf}
\end{center}
\caption{The \locate{} performance of \SSA{} and \RFM{} on NA12878 relative to
the human reference genome without chromosome~Y. Index size in bits per
character for various \SA{} sample intervals, and the time required to perform
2~million queries of length $32$ with a total of 255~million
occurrences.}\label{fig:locate}
\end{figure}

\subsection{Synthetic collections}

In order to determine how the differences between the reference sequence and
the target sequence affect the size of relative structures, we built \RCST{}
for various \emph{synthetic datasets}. We took the human reference genome as
the reference sequence, and generated synthetic target sequences with
\emph{mutation rates} $p \in \set{0.0001, 0.0003, 0.001, 0.003, 0.01, 0.03,
0.1}$. A total of 90\% of the mutations were single-character substitutions,
while 5\% were insertions and another 5\% deletions. The length of an
insertion or deletion was $k \ge 1$ with probability $0.2 \cdot 0.8^{k-1}$.

The results can be seen in Figure~\ref{fig:synthetic}~(left). The \RLCP{}
array quickly grew with increasing mutation rates, peaking out at $p = 0.01$.
At that point, the average length of an \RLZ{} phrase was comparable to what
could be found in the \DLCP{} arrays of unrelated DNA sequences. With even
higher mutation rates, the phrases became slightly longer due to the smaller
average \LCP{} values. The \RFM{} index, on the other hand, remained small
until $p = 0.003$. Afterwards, the index started to grow quickly, eventually
overtaking the \RLCP{} array.

\begin{figure}
\begin{center}
\includegraphics{synth.pdf}\hspace{-0.4in}\includegraphics{comp.pdf}
\end{center}
\caption{Index size in bits per character vs.~mutation rate for synthetic
datasets. Left: Synthetic genomes relative to the human reference genome.
Right: Collections of 25 synthetic sequences relative to a 20~MB
reference.}\label{fig:synthetic}
\end{figure}

We also compared the size of the relative \CST{} to a compressed suffix tree
for repetitive collections. While the structures are intended for different
purposes, the comparison shows how much additional space is needed to provide
access to the compressed suffix trees of individual datasets. We chose to skip
the \CSTnpr{} for repetitive collections \cite{Abeliuk2013}, as its
implementation was not stable enough. Because the implementation of \CSTsada{}
for repetitive collections (\GCT) \cite{Navarro2014} is based on a library
that uses signed 32\nobreakdash-bit integers internally, we had to limit the
size of the collections to about 500 megabytes for this experiment. We
therefore took a 20\nobreakdash-megabyte prefix of the human reference genome,
and generated 25 synthetic sequences for each mutation rate (see above).

Figure~\ref{fig:synthetic}~(right) shows the sizes of the compressed suffix
trees. The numbers for \RCST{} include individual indexes for each of the 25
target sequences as well as the reference data, while the numbers for \GCT{}
are for a single index containing the 25 sequences. With low mutation rates,
\RCST{} was not much larger than \GCT{}. The size of \RCST{} starts growing
quickly at around $p = 0.001$, while the size of \GCT{} stabilizes at
3\nobreakdash--4~bpc.

\subsection{Suffix tree operations}

In the final set of experiments, we compared the performance of \RCST{} to the
SDSL implementations of various compressed suffix trees. We used the maternal
haplotypes of NA12878 as the target sequence and the human reference genome
without chromosome~Y as the reference sequence. We then built \RCST, \CSTsada,
\CSTnpr, and \FCST{} for the target sequence. \CSTsada{} used \emph{Sadakane's
compressed suffix array} (\CSAsada) \cite{Sadakane2003} as its \CSA, while the
other SDSL implementations used \SSA. All SDSL compressed suffix trees used
\PLCP{} as their \LCP{} encoding, but we also built \CSTnpr{} with \LCPbyte.

We used two algorithms for the performance comparison. The first algorithm was
\emph{depth-first traversal} of the suffix tree. We used SDSL iterators
(\texttt{cst\_dfs\_const\_forward\_iterator}), which in turn used operations
$\mRoot$, $\mLeaf$, $\mParent$, $\mFChild$, and $\mNSibling$. The traversal
was generally quite fast, because the iterators cached the most recent parent
nodes.

The second algorithm was computing \emph{matching statistics}
\cite{Chang1994}. Given sequence $S'$ of length $n'$, the goal was to find the
longest prefix $S'[i,i+\ell_{i}-1]$ of each suffix $S'[i,n']$ occurring in
sequence $S$. For each such prefix, we store its length $\ell_{i}$ and the
suffix array range $\mSA(S)[sp_{i},ep_{i}]$ of its occurrences in sequence
$S$. We computed the matching statistics with forward searching, using
operations $\mRoot$, $\mSDepth$, $\mSLink$, $\mChild$, and $\mLetter$.
Computing the matching statistics would probably have been faster with
backward searching \cite{Ohlebusch2010a}, but the purpose of this experiment
was to test a different part of the interface.

We used the \emph{paternal haplotypes} of NA12878 as sequence $S'$. Because
forward searching is much slower than tree traversal, we only computed
matching statistics for chromosome~1. We also truncated the runs of character
$N$ in sequence $S'$ into a single character. Because the time complexities of
certain operations in the succinct tree representation used in SDSL depend on
the depth of the current node, including the runs (which make the suffix tree
extremely deep locally) would have made the SDSL suffix trees much slower than
\RCST.

The results can be seen in Table~\ref{table:cst}. \RCST{} was clearly smaller
than \FCST, and several times smaller than the other compressed suffix trees.
In depth-first traversal, \RCST{} was 2.2~times slower than \CSTnpr{} and
about 8~times slower than \CSTsada. For computing matching statistics, \RCST{}
was 2.9~times slower than \CSTsada{} and 4.7\nobreakdash--7.6~times slower
than \CSTnpr{}. With the optional \rselect{} structure, the differences were
reduced to 1.2~times and 2.0\nobreakdash--3.2~times, respectively. We did not
run the full experiments with \FCST, because it was much slower than the rest.
According to earlier results, \FCST{} is about two orders of magnitude slower
than \CSTsada{} and \CSTnpr{} \cite{Abeliuk2013}.

\begin{table}
\caption{Compressed suffix trees for the maternal haplotypes of NA12878
relative to the human reference genome without chromosome~Y. Component
choices, index size in bits per character, and time in minutes for depth-first
traversal and computing matching statistics for the paternal haplotypes of
chromosome~1 of NA12878.}\label{table:cst}
\setlength{\extrarowheight}{2pt}
\setlength{\tabcolsep}{3pt}
\begin{center}
\begin{tabular}{c|cc|c|c|c}
\hline
\textbf{\CST} & \textbf{\CSA} & \textbf{\LCP} & \textbf{Size} &
\textbf{Traversal} & \textbf{Matching statistics} \\
\hline
\CSTsada           & \CSAsada & \PLCP    & 12.33 bpc &  5 min & 315 min \\
\CSTnpr            & \SSA     & \PLCP    & 10.79 bpc & 18 min & 195 min \\
\CSTnpr            & \SSA     & \LCPbyte & 18.08 bpc & 18 min & 120 min \\
\FCST              & \SSA     & \PLCP    &  4.98 bpc &     -- &      -- \\
\RCST              & \RFM     & \RLCP    &  3.16 bpc & 39 min & 910 min \\
\RCST{} + \rselect & \RFM     & \RLCP    &  3.61 bpc & 39 min & 389 min \\
\hline
\end{tabular}
\end{center}
\end{table}


\section{Conclusions}\label{section:conclusions}

\begin{itemize}
\item conclusions
\item other ideas: RLZ bitvectors, RLZ pointer compression
\item analyze the similarity between LCP arrays based on edit distance
\item Sada-RLZ: RLZ compression of tree topology and bitvector H of PLCP
\end{itemize}

%\newpage %for ALENEX format

\bibliographystyle{plain}
\bibliography{rcst}


\end{document}
