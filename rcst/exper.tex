
\section{Experiments}

We have implemented the relative compressed suffix tree in C++, extending the
old relative FM-index implementation.\footnote{The implementation is available
at \url{https://github.com/jltsiren/relative-fm}.} The implementation is based
on the \emph{Succinct Data Structure Library (SDSL) 2.0}~\cite{Gog2014b}. Some
parts of the implementation have been parallelized using \emph{OpenMP} and the
\emph{libstdc++ parallel mode}.

We used the SDSL implementation of the \emph{succinct suffix array} (\SSA{})
\cite{Ferragina2007a,Maekinen2005} as our reference \CSA{} and our baseline
index. \SSA{} encodes the Burrows-Wheeler transform as a \emph{Huffman-shaped
wavelet tree}, combining very fast queries with size close to the
\emph{order\nobreakdash-$0$ empirical entropy}. These properties make it the
index of choice for DNA sequences \cite{Ferragina2009a}. Due to the long runs
of character $N$, \emph{fixed block compression boosting}
\cite{Kaerkkaeinen2011} could reduce index size without significantly
increasing query times. Unfortunately there is no implementation capable of
handling multi-gigabyte datasets available.

We sampled \SA{} in suffix order and \ISA{} in text order. In \SSA, the sample
intervals were $17$ for \SA{} and $64$ for \ISA. In \RFM, we used sample
interval $257$ for \SA{} and $512$ for \ISA{} to handle the regions that do
not exist in the reference. The sample intervals for suffix order sampling
were primes due to the long runs of character $N$ in the assembled genomes. If
the number of long runs of character $N$ in the indexed sequence is even, the
lexicographic ranks of almost all suffixes in half of the runs are odd, and
those runs are almost completely unsampled. This can be avoided by making the
sample interval and the number of runs \emph{relatively prime}.

The experiments were run on a computer cluster running LSF 9.1.1.1 on Ubuntu
12.04 with Linux kernel 2.6.32. For most experiments, we used cluster nodes
with two 16-core AMD Opteron 6378 processors and 256 gigabytes of memory. Some
index construction jobs may have run on nodes with two 12-core AMD Opteron
6174 processors and 80 or 128 gigabytes of memory. All query experiments were
run single-threaded with no other jobs in the same node. Index construction
used 8 parallel threads, but there may have been other jobs running on the
same nodes at the same time.

As our primary target sequence, we used the \emph{maternal haplotypes} of the
\emph{1000 Genomes Project individual NA12878}~\cite{Rozowsky2011}. As the
target sequence, we used the 1000 Genomes Project version of the \emph{GRCh37
assembly} of the \emph{human reference
genome}.\footnote{\url{ftp://ftp.1000genomes.ebi.ac.uk/vol1/ftp/technical/reference/}}
Because NA12878 is female, we also created a reference sequence without the
chromosome~Y.

In the following, a basic FM-index is an index supporting only \find{}
queries, while a full index also supports \locate{} and \extract{} queries.

\subsection{Indexes and their sizes}

Table~\ref{table:construction} lists the resource requirements for building
the relative indexes, assuming that we have already built the corresponding
non-relative structures for the sequences. As a comparison, building an
FM-index for a human genome typically takes 16--17 minutes and 25--26
gigabytes of memory. While the construction of the basic \RFM{} index is
highly optimized, the other construction algorithms are just the first
implementations.

The construction times for the relative \CST{} do not include the time
required for indexing the \DLCP{} array of the reference sequence. While this
takes another two hours, it only needs to be done once for every reference
sequence. Building the optional \rselect{} structures takes 9--10 minutes and
around $\abs{R} + \abs{S}$ bits of working space in addition to \RFM{} and
\rselect.

\begin{table}
\caption{Sequence lengths and resources used by index construction for NA12878
relative to the human reference genome with and without chromosome~Y. Approx
and Inv denote the approximate \LCS{} and the bwt-invariant subsequence.
Sequence lengths are in millions of base pairs, while construction resources
are in minutes of wall clock time and gigabytes of
memory.}\label{table:construction}
\setlength{\extrarowheight}{2pt}
\setlength{\tabcolsep}{3pt}
\begin{center}
\begin{tabular}{c|cccc|cc|cc|cc}
\hline
 &
\multicolumn{4}{c|}{\textbf{Sequence length}} &
\multicolumn{2}{c|}{\textbf{\RFM{} (basic)}} &
\multicolumn{2}{c|}{\textbf{\RFM{} (full)}} &
\multicolumn{2}{c}{\textbf{\RCST}} \\
\textbf{ChrY} &
\textbf{Reference} & \textbf{Target} & \textbf{Approx} & \textbf{Inv} &
\textbf{Time} & \textbf{Space} &
\textbf{Time} & \textbf{Space} &
\textbf{Time} & \textbf{Space} \\
\hline
yes & 3096M & 3036M & 2992M & 2980M & 2.35 min & 4.96 GB & 238 min & 83.7 GB &
379 min & 99.0 GB \\
no  & 3036M & 3036M & 2991M & 2980M & 2.28 min & 4.86 GB & 214 min & 82.3 GB &
398 min & 97.2 GB \\
\hline
\end{tabular}
\end{center}
\end{table}

The sizes of the final indexes are listed in Table~\ref{table:indexes}. While
the basic \RFM{} index is 5\nobreakdash--6 times smaller than the basic \SSA, the
full \RFM{} is 4.4\nobreakdash--5 times smaller than the full \SSA. The
\RLCP{} array is about twice as large as the full \RFM{} index, increasing the
total size of the \RCST{} to 3.2\nobreakdash--4.3 bits per character. The
optional relative select structure is almost as large as the basic \RFM{}
index. As the relative structures are significantly larger relative to a male
reference than relative to a female reference, keeping a separate female
reference seems worthwhile, if there are more than a few female genomes among
the target sequences.

\begin{table}
\caption{Various indexes for NA12878 relative to the human reference genome
with and without chromosome~Y. Index sizes are in megabytes and in bits per
character.}\label{table:indexes}
\setlength{\extrarowheight}{2pt}
\setlength{\tabcolsep}{3pt}
\begin{center}
\begin{tabular}{c|cc|cc|cccc}
\hline
 &
\multicolumn{2}{c|}{\textbf{\SSA}} &
\multicolumn{2}{c|}{\textbf{\RFM}} &
\multicolumn{4}{c}{\textbf{\RCST}} \\
\textbf{ChrY} &
\textbf{Basic} & \textbf{Full} &
\textbf{Basic} & \textbf{Full} &
\textbf{\RFM} & \textbf{\RLCP} & \textbf{Total} & \textbf{\rselect} \\
\hline
\multirow{2}{*}{yes} &  1090 MB &  1953 MB &   218 MB &   447 MB &   447 MB &
1100 MB &  1547 MB &   190 MB \\
                     & 3.01 bpc & 5.42 bpc & 0.60 bpc & 1.23 bpc & 1.23 bpc &
3.04 bpc & 4.27 bpc & 0.52 bpc \\
\hline
\multirow{2}{*}{no}  &  1090 MB &  1953 MB &   181 MB &   395 MB &   395 MB &
750 MB &  1145 MB &   163 MB \\
                     & 3.01 bpc & 5.42 bpc & 0.50 bpc & 1.09 bpc & 1.09 bpc &
2.07 bpc & 3.16 bpc & 0.45 bpc \\
\hline
\end{tabular}
\end{center}
\end{table}

Tables~\ref{table:rfm components} and~\ref{table:rlcp components} list
the sizes of the individual components of the relative FM-index and the
\RLCP{} array. Including the chromosome~Y in the reference increases the sizes
of almost all relative components, with the exception of $\mCS(\mBWT_{S})$ and
$\mLCS(R,S)$. In the first case, the common subsequence still covers
approximately the same positions in $\mBWT_{S}$ as before. In the second case,
chromosome~Y appears in bitvector $B_{R}$ as a long run of \zerobit{}s, which
compresses well. The components of a full \RFM{} index are slightly larger
than the corresponding components of a basic \RFM{} index, because the
bwt-invariant subsequence is slightly shorter than the approximate longest
common subsequence (see Table~\ref{table:construction}).

\begin{table}
\caption{Breakdown of component sizes in the \RFM{} index for NA12878 relative
to the human reference genome with and without chromosome~Y in bits per
character.}\label{table:rfm components}
\setlength{\extrarowheight}{2pt}
\setlength{\tabcolsep}{3pt}
\begin{center}
\begin{tabular}{c|cc|cc}
\hline
 & \multicolumn{2}{c|}{\textbf{Basic \RFM}} & \multicolumn{2}{c}{\textbf{Full
\RFM}} \\
\textbf{ChrY} & \textbf{yes} & \textbf{no} & \textbf{yes} & \textbf{no} \\
\hline
\textbf{\RFM}              & \textbf{0.60 bpc} & \textbf{0.50 bpc} &
\textbf{1.23 bpc} & \textbf{1.09 bpc} \\
$\mCS(\mBWT_{R})$           &          0.11 bpc &          0.04 bpc &
0.12 bpc &          0.05 bpc \\
$\mCS(\mBWT_{S})$           &          0.04 bpc &          0.04 bpc &
0.05 bpc &          0.05 bpc \\
$\mLCS(\mBWT_{R},\mBWT_{S})$ &          0.45 bpc &          0.42 bpc &
0.52 bpc &          0.45 bpc \\
$\mLCS(R,S)$               &                -- &                -- &
0.35 bpc &          0.35 bpc \\
\SA{} samples              &                -- &                -- &
0.12 bpc &          0.12 bpc \\
\ISA{} samples             &                -- &                -- &
0.06 bpc &          0.06 bpc \\
\hline
\end{tabular}
\end{center}
\end{table}

\begin{table}
\caption{Breakdown of component sizes in the \RLCP{} array for NA12878
relative to the human reference genome with and without chromosome~Y in bits
per character. The components are phrase pointers, phrase boundaries in the
target \LCP{} array, mismatch characters, and the minima
tree.}\label{table:rlcp components}
\setlength{\extrarowheight}{2pt}
\setlength{\tabcolsep}{3pt}
\begin{center}
\begin{tabular}{c|cccc|c}
\hline
\textbf{ChrY} & $\mathbf{W_{p}}$  & $\mathbf{W_{\ell}}$ & $\mathbf{W_{c}}$ &
$\mathbf{M_{\mLCP}}$ & \textbf{Total} \\
\hline
yes & 1.44 bpc & 0.34 bpc & 0.54 bpc & 0.71 bpc & 3.04 bpc \\
no  & 1.11 bpc & 0.27 bpc & 0.30 bpc & 0.39 bpc & 2.07 bpc \\
\hline
\end{tabular}
\end{center}
\end{table}

\subsection{Query times}

Average query times for the basic operations can be seen in Tables~\ref{table:rfm
queries} and~\ref{table:rlcp queries}. For \LF{} and \Psiop{} queries
with the full \SSA{}, the full \RFM, and the full \RFM{} augmented with
\rselect, the results are similar to the earlier ones with basic indexes
\cite{Boucher2015}. Random access to the \RLCP{} array is about 20 times
slower than to the \LCP{} array. The \LCP{} array provides sequential access
iterators, which are much faster than using random access sequentially. The
\RLCP{} array does not have such iterators, because subsequent phrases are
often copied from different parts of the reference. However, queries based on
the minima tree (\nsv, \psv, and \rmq) use fast sequential access to the
\RLCP{} array inside a phrase. \SSA{} and the \LCP{} array are consistently
slower when the reference does not contain chromosome~Y, even though the
structures are identical in either case. This is probably a memory management
artifact that depends on other memory allocations.

\begin{table}
\caption{Query times with \SSA{} and the \RFM{} index for NA12878 relative to
the human reference genome with and without chromosome~Y in microseconds. The
query times are averages over 10~million random queries.}\label{table:rfm
queries}
\setlength{\extrarowheight}{2pt}
\setlength{\tabcolsep}{3pt}
\begin{center}
\begin{tabular}{c|cc|cc|c}
\hline
 & \multicolumn{2}{c|}{\textbf{\SSA}} & \multicolumn{2}{c|}{\textbf{\RFM}} &
\textbf{\rselect} \\
\textbf{ChrY} & \textbf{\LF} & \textbf{\Psiop} & \textbf{\LF} &
\textbf{\Psiop} & \textbf{\Psiop} \\
\hline
yes & 0.560 \mus & 1.139 \mus & 3.980 \mus & 47.249 \mus & 6.277 \mus \\
no  & 0.627 \mus & 1.563 \mus & 3.861 \mus & 55.068 \mus & 6.486 \mus \\
\hline
\end{tabular}
\end{center}
\end{table}

\begin{table}
\caption{Query times with the reference \LCP{} array and the \RLCP{} array for
NA12878 relative to the human reference genome with and without chromosome~Y
in microseconds. For the random queries, the query times are averages over
100~million queries. The range lengths for the \rmq{} queries were $16^{k}$ (for
$k \ge 1$) with probability $0.5^{k}$. For sequential access, the times are
averages per position for scanning the entire \LCP{} array.}\label{table:rlcp
queries}
\setlength{\extrarowheight}{2pt}
\setlength{\tabcolsep}{3pt}
\begin{center}
\begin{tabular}{c|cc|ccccc}
\hline
 & \multicolumn{2}{c|}{\textbf{\LCP{} array}} &
\multicolumn{5}{c}{\textbf{\RLCP{} array}} \\
\textbf{ChrY} & \textbf{Random} & \textbf{Sequential} & \textbf{Random} &
\textbf{Sequential} & \textbf{\nsv} & \textbf{\psv} & \textbf{\rmq} \\
\hline
yes & 0.052 \mus & 0.001 \mus & 1.096 \mus & 0.119 \mus & 1.910 \mus & 1.935
\mus & 2.769 \mus \\
no  & 0.070 \mus & 0.001 \mus & 1.263 \mus & 0.124 \mus & 1.801 \mus & 1.923
\mus & 2.605 \mus \\
\hline
\end{tabular}
\end{center}
\end{table}

We also tested the \locate{} performance of the full \RFM{} index, and
compared it to \SSA. We built \SSA{} with \SA{} sample intervals $7$, $17$,
$31$, $61$, and $127$ for the reference and the target sequence, using only
the reference without chromosome~Y. \ISA{} sample interval was set to the
maximum of $64$ and the \SA{} sample interval. We then built \RFM{} for the
target sequence, and extracted 2~million random patterns of length $32$,
consisting of characters $ACGT$, from the target sequence. The time/space
trade-offs for \locate{} queries with these patterns can be seen in
Figure~\ref{fig:locate}. While the \RFM{} index was 8.6x slower than \SSA{}
with sample interval $7$, the absolute performance difference remained almost
constant with longer sample intervals. With sample interval $127$, \RFM{} was
only 1.2x slower than \SSA.

\begin{figure}
\begin{center}
\includegraphics{locate.pdf}
\end{center}
\caption{The \locate{} performance of \SSA{} and \RFM{} on NA12878 relative to
the human reference genome without chromosome~Y. Index size in bits per
character for various \SA{} sample intervals, and the time required to perform
2~million queries of length $32$ with a total of 255~million
occurrences.}\label{fig:locate}
\end{figure}

\subsection{Synthetic collections}

In order to determine how the differences between the reference sequence and
the target sequence affect the size of relative structures, we built \RCST{}
for various \emph{synthetic datasets}. We took the human reference genome as
the reference sequence, and generated synthetic target sequences with
\emph{mutation rates} $p \in \set{0.0001, 0.0003, 0.001, 0.003, 0.01, 0.03,
0.1}$. A total of 90\% of the mutations were single-character substitutions,
while 5\% were insertions and another 5\% deletions. The length of an
insertion or deletion was $k \ge 1$ with probability $0.2 \cdot 0.8^{k-1}$.

The results can be seen in Figure~\ref{fig:synthetic}~(left). The \RLCP{}
array quickly grew with increasing mutation rates, peaking out at $p = 0.01$.
At that point, the average length of an \RLZ{} phrase was comparable to what
could be found in the \DLCP{} arrays of unrelated DNA sequences. With even
higher mutation rates, the phrases became slightly longer due to the smaller
average \LCP{} values. The \RFM{} index, on the other hand, remained small
until $p = 0.003$. Afterwards, the index started to grow quickly, eventually
overtaking the \RLCP{} array.

\begin{figure}
\begin{center}
\includegraphics{synth.pdf}\hspace{-0.4in}\includegraphics{comp.pdf}
\end{center}
\caption{Index size in bits per character vs.~mutation rate for synthetic
datasets. Left: Synthetic genomes relative to the human reference genome.
Right: Collections of 25 synthetic sequences relative to a 20~MB
reference.}\label{fig:synthetic}
\end{figure}

We also compared the size of the relative \CST{} to a compressed suffix tree
for repetitive collections. While the structures are intended for different
purposes, the comparison shows how much additional space is needed to provide
access to the compressed suffix trees of individual datasets. We chose to skip
the \CSTnpr{} for repetitive collections \cite{Abeliuk2013}, as its
implementation was not stable enough. Because the implementation of \CSTsada{}
for repetitive collections (\GCT) \cite{Navarro2014} is based on a library
that uses signed 32\nobreakdash-bit integers internally, we had to limit the
size of the collections to about 500 megabytes for this experiment. We
therefore took a 20\nobreakdash-megabyte prefix of the human reference genome,
and generated 25 synthetic sequences for each mutation rate (see above).

Figure~\ref{fig:synthetic}~(right) shows the sizes of the compressed suffix
trees. The numbers for \RCST{} include individual indexes for each of the 25
target sequences as well as the reference data, while the numbers for \GCT{}
are for a single index containing the 25 sequences. With low mutation rates,
\RCST{} was not much larger than \GCT{}. The size of \RCST{} starts growing
quickly at around $p = 0.001$, while the size of \GCT{} stabilizes at
3\nobreakdash--4~bpc.

\subsection{Suffix tree operations}

In the final set of experiments, we compared the performance of \RCST{} to the
SDSL implementations of various compressed suffix trees. We used the maternal
haplotypes of NA12878 as the target sequence and the human reference genome
without chromosome~Y as the reference sequence. We then built \RCST, \CSTsada,
\CSTnpr, and \FCST{} for the target sequence. \CSTsada{} used \emph{Sadakane's
compressed suffix array} (\CSAsada) \cite{Sadakane2003} as its \CSA, while the
other SDSL implementations used \SSA. All SDSL compressed suffix trees used
\PLCP{} as their \LCP{} encoding, but we also built \CSTnpr{} with \LCPbyte.

We used two algorithms for the performance comparison. The first algorithm was
\emph{depth-first traversal} of the suffix tree. We used SDSL iterators
(\texttt{cst\_dfs\_const\_forward\_iterator}), which in turn used operations
$\mRoot$, $\mLeaf$, $\mParent$, $\mFChild$, and $\mNSibling$. The traversal
was generally quite fast, because the iterators cached the most recent parent
nodes.

The second algorithm was computing \emph{matching statistics}
\cite{Chang1994}. Given sequence $S'$ of length $n'$, the goal was to find the
longest prefix $S'[i,i+\ell_{i}-1]$ of each suffix $S'[i,n']$ occurring in
sequence $S$. For each such prefix, we store its length $\ell_{i}$ and the
suffix array range $\mSA_{S}[sp_{i},ep_{i}]$ of its occurrences in sequence
$S$. We computed the matching statistics with forward searching, using
operations $\mRoot$, $\mSDepth$, $\mSLink$, $\mChild$, and $\mLetter$.
Computing the matching statistics would probably have been faster with
backward searching \cite{Ohlebusch2010a}, but the purpose of this experiment
was to test a different part of the interface.

We used the \emph{paternal haplotypes} of NA12878 as sequence $S'$. Because
forward searching is much slower than tree traversal, we only computed
matching statistics for chromosome~1. We also truncated the runs of character
$N$ in sequence $S'$ into a single character. Because the time complexities of
certain operations in the succinct tree representation used in SDSL depend on
the depth of the current node, including the runs (which make the suffix tree
extremely deep locally) would have made the SDSL suffix trees much slower than
\RCST.

The results can be seen in Table~\ref{table:cst}. \RCST{} was clearly smaller
than \FCST, and several times smaller than the other compressed suffix trees.
In depth-first traversal, \RCST{} was 2.2~times slower than \CSTnpr{} and
about 8~times slower than \CSTsada. For computing matching statistics, \RCST{}
was 2.9~times slower than \CSTsada{} and 4.7\nobreakdash--7.6~times slower
than \CSTnpr{}. With the optional \rselect{} structure, the differences were
reduced to 1.2~times and 2.0\nobreakdash--3.2~times, respectively. We did not
run the full experiments with \FCST, because it was much slower than the rest.
According to earlier results, \FCST{} is about two orders of magnitude slower
than \CSTsada{} and \CSTnpr{} \cite{Abeliuk2013}.

\begin{table}
\caption{Compressed suffix trees for the maternal haplotypes of NA12878
relative to the human reference genome without chromosome~Y. Component
choices, index size in bits per character, and time in minutes for depth-first
traversal and computing matching statistics for the paternal haplotypes of
chromosome~1 of NA12878.}\label{table:cst}
\setlength{\extrarowheight}{2pt}
\setlength{\tabcolsep}{3pt}
\begin{center}
\begin{tabular}{c|cc|c|c|c}
\hline
\textbf{\CST} & \textbf{\CSA} & \textbf{\LCP} & \textbf{Size} &
\textbf{Traversal} & \textbf{Matching statistics} \\
\hline
\CSTsada           & \CSAsada & \PLCP    & 12.33 bpc &  5 min & 315 min \\
\CSTnpr            & \SSA     & \PLCP    & 10.79 bpc & 18 min & 195 min \\
\CSTnpr            & \SSA     & \LCPbyte & 18.08 bpc & 18 min & 120 min \\
\FCST              & \SSA     & --       &  4.98 bpc &     -- &      -- \\
\RCST              & \RFM     & \RLCP    &  3.16 bpc & 39 min & 910 min \\
\RCST{} + \rselect & \RFM     & \RLCP    &  3.61 bpc & 39 min & 389 min \\
\hline
\end{tabular}
\end{center}
\end{table}


