
\section{Background}

A \emph{string} $S[1,n] = s_{1} \dotso s_{n}$ is a sequence of
\emph{characters} over an \emph{alphabet} $\Sigma = \set{1, \dotsc, \sigma}$.
For indexing purposes, we often consider \emph{text} strings $T[1,n]$ that are
terminated by an \emph{endmarker} $T[n] = \$ = 0$ not occurring elsewhere in
the text. \emph{Binary} sequences are sequences over the alphabet $\set{0,1}$.
If $B[1,n]$ is a binary sequence, its \emph{complement} is binary sequence
$\complement{B}[1,n]$, with $\complement{B}[i] = 1 - B[i]$.

For any binary sequence $B[1,n]$, we define the \emph{subsequence} $S[B]$ of
string $S[1,n]$ as the concatenation of the characters $s_{i}$ with $B[i] = 1$.
The complement $\complement{S}[B]$ of subsequence $S[B]$ is the subsequence
$S[\complement{B}]$. Contiguous subsequences $S[i,j]$ are called
\emph{substrings}. Substrings of the form $S[1,j]$ and $S[i,n]$, $i,j \in
[1,n]$, are called \emph{prefixes} and \emph{suffixes}, respectively. We
define the \emph{lexicographic order} among strings in the usual way.

\subsection{Full-text indexes}

The \emph{suffix tree} (\ST)~\cite{Weiner1973} of text $T$ is a trie
containing the suffixes of $T$, with unary paths compacted into single edges.
Because the degree of every internal node is at least two, there can be at most
$2n-1$ nodes, and the suffix tree can be stored in $\Oh(n \log n)$ bits. In
practice, this is at least $10n$ bytes for small texts~\cite{Kurtz1999}, and
more for large texts as the pointers grow larger. If $v$ is a node of a suffix
tree, we write $\pi(v)$ to denote the concatenation of the labels of the path
from the root to node $v$.

\emph{Suffix arrays} (\SA)~\cite{Manber1993} were introduced as a
space-efficient alternative to suffix trees. The suffix array $\mSA_{T}[1,n]$ of
text $T$ is an array of pointers to the suffixes of the text in lexicographic
order.\footnote{We drop the subscript, if the text is evident from the context.}
In its basic form, the suffix array requires $n \log n$ bits in
addition to the text, but its functionality is more limited than that of the
suffix tree. In addition to the suffix array, many algorithms also use the
\emph{inverse suffix array} $\mISA[1,n]$, with $\mSA[\mISA[i]] = i$ for all
$i$.

Let $\mlcp(S_{1}, S_{2})$ be the length of the \emph{longest common prefix}
(\LCP) of strings $S_{1}$ and $S_{2}$. The \LCP{}
\emph{array}~\cite{Manber1993} $\mLCP[1,n]$ of text $T$ stores the \LCP{}
lengths for lexicographically adjacent suffixes of $T$ as $\mLCP[i] =
\mlcp(T[\mSA[i-1],n], T[\mSA[i],n])$. Let $v$ be an internal node of the
suffix tree, $\ell = \abs{\pi(v)}$ the \emph{string depth} of node $v$, and
$\mSA[sp,ep]$ the corresponding suffix array interval. The following
properties hold for the \emph{lcp-interval} $\mLCP[sp,ep]$: i) $\mLCP[sp] <
\ell$; ii) $\mLCP[i] \ge \ell$ for all $sp < i \le ep$; iii) $\mLCP[i] = \ell$
for at least one $sp < i \le ep$; and iv) $\mLCP[ep+1] <
\ell$~\cite{Abouelhoda2004}.

Abouelhoda et al.~\cite{Abouelhoda2004} showed how traversals on the suffix
tree could be simulated using the suffix array, the \LCP{} array, and a
representation of the suffix tree topology based on lcp-intervals, paving
the way for more space-efficient suffix tree representations.

\subsection{Compressed text indexes}

Data structures supporting \rank{} and \select{} queries over sequences are
the main building blocks of compressed text indexes. If $S$ is a sequence, we
define $\mrank_{c}(S,i)$ to be the number of occurrences of character $c$ in
the prefix $S[1,i]$, while $\mselect_{c}(S,j)$ is the position of the occurrence
of rank $j$ in sequence $S$. A \emph{bitvector} is a representation of a
binary sequence $B$ supporting fast \rank{} and \select{} queries.
\emph{Wavelet trees} (\WT)~\cite{Grossi2003} use bitvectors to support \rank{}
and \select{} on general sequences.

The \emph{Burrows-Wheeler transform} (\BWT)~\cite{Burrows1994} is a reversible
permutation $\mBWT[1,n]$ of text $T$. It is defined as $\mBWT[i] = T[\mSA[i] -
1]$ (with $\mBWT[i] = T[n]$, if $\SA[i] = 1$). Originally intended for data
compression, the Burrows-Wheeler transform has been widely used in
space-efficient text indexes, because it shares the combinatorial structure of
the suffix tree and the suffix array.

Let \LF{} be a function such that $\mSA[\mLF(i)] = \mSA[i] - 1$ (with
$\mSA[\mLF(i)] = n$, if $\mSA[i] = 1$). We can compute it as $\mLF(i) =
\mC[\mBWT[i]] + \mrank_{\mBWT[i]}(\mBWT, i)$, where $\mC[c]$ is the number of
occurrences of characters with lexicographical values smaller than $c$ in
\BWT. The inverse function of \LF{} is $\mPsi$, with $\mPsi(i) =
\mselect_{c}(\mBWT, i - \mC[c])$, where $c$ is the largest character value
with $\mC[c] < i$. With functions $\mPsi$ and \LF, we can move forward and
backward in the text, while maintaining the lexicographic rank of the current
suffix. If the sequence $S$ is not evident from the context, we write $\mLF_{S}$
and $\mPsi_{S}$.

\emph{Compressed suffix arrays} (\CSA) \cite{Ferragina2005a,Grossi2005} are
text indexes supporting similar functionality to the suffix array. This
includes the following queries: i) $\mfind(P) = [sp,ep]$ determines the
lexicographic range of suffixes starting with \emph{pattern} $P[1,\ell]$; ii)
$\mlocate(sp,ep) = \mSA[sp,ep]$ returns the starting positions of these
suffixes; and iii) $\mextract(i,j) = T[i,j]$ extracts substrings of the text.
In practice, the \find{} performance of compressed suffix arrays can be
competitive with suffix arrays, while \locate{} queries are orders of
magnitude slower~\cite{Ferragina2009a}. Typical index sizes are less than the
size of the uncompressed text.

The \emph{FM-index} (\FMI) \cite{Ferragina2005a} is a common type of
compressed suffix array. A typical implementation stores the \BWT{} in a
wavelet tree \cite{Grossi2003}. The index implements \find{} queries via a
process called \emph{backward search}. Let $[sp,ep]$ be the lexicographic
range of the suffixes of the text that start with suffix $P[i+1,\ell]$ of the
pattern. We can find the range matching suffix $P[i,\ell]$ with a
generalization of function \LF{} as
$$
\mLF([sp,ep],P[i]) =
[\mC[P[i]] + \mrank_{P[i]}(\mBWT, sp-1) + 1,
\mC[P[i]] + \mrank_{P[i]}(\mBWT, ep)].
$$

We support \locate{} queries by \emph{sampling} some suffix array pointers. If
we want to determine a value $\mSA[i]$ that has not been sampled, we can
compute it as $\mSA[i] = \mSA[j]+k$, where $\mSA[j]$ is a sampled pointer
found by iterating \LF{} $k$ times, starting from position $i$. Given
\emph{sample interval} $d$, the samples can be chosen in \emph{suffix order},
sampling $\mSA[i]$ at positions divisible by $d$, or in \emph{text order},
sampling $T[i]$ at positions divisible by $d$ and marking the sampled \SA{}
positions in a bitvector. Suffix-order sampling requires less space, often
resulting in better time/space trade-offs in practice, while text-order
sampling guarantees better worst-case performance. We also sample some \ISA{}
pointers for \extract{} queries. To extract $T[i,j]$, we find the nearest
sampled pointer after $\mISA[j]$, and traverse backwards to $T[i]$ with
function \LF.

\begin{table}
\centering{}
\caption{Typical compressed suffix tree operations.}\label{table:cst
operations}

\begin{tabular}{ll}
\hline
\noalign{\smallskip}
\textbf{Operation}  & \textbf{Description} \\
\noalign{\smallskip}
\hline
\noalign{\smallskip}
$\mRoot()$          & The root of the tree. \\
$\mLeaf(v)$         & Is node $v$ a leaf? \\
$\mAncestor(v,w)$   & Is node $v$ an ancestor of node $w$? \\
\noalign{\smallskip}
$\mCount(v)$        & Number of leaves in the subtree with $v$ as the root. \\
$\mLocate(v)$       & Pointer to the suffix corresponding to leaf $v$. \\
\noalign{\smallskip}
$\mParent(v)$       & The parent of node $v$. \\
$\mFChild(v)$       & The first child of node $v$ in alphabetic order. \\
$\mNSibling(v)$     & The next sibling of node $v$ in alphabetic order. \\
$\mLCA(v,w)$        & The lowest common ancestor of nodes $v$ and $w$. \\
\noalign{\smallskip}
$\mSDepth(v)$       & \emph{String depth}: Length $\ell = \abs{\pi(v)}$ of the
label from the root to node $v$. \\
$\mTDepth(v)$       & \emph{Tree depth}: The depth of node $v$ in the suffix
tree. \\
$\mLAQ_{S}(v,d)$    & The highest ancestor of node $v$ with string depth at
least $d$. \\
$\mLAQ_{T}(v,d)$    & The ancestor of node $v$ with tree depth $d$. \\
\noalign{\smallskip}
$\mSLink(v)$        & \emph{Suffix link}: Node $w$ such that $\pi(v) = c \pi(w)$ for
a character $c \in \Sigma$. \\
$\mSLink^{k}(v)$    & Suffix link iterated $k$ times. \\
\noalign{\smallskip}
$\mChild(v,c)$      & The child of node $v$ with edge label starting with
character $c$. \\
$\mLetter(v,i)$     & The character $\pi(v)[i]$. \\
\noalign{\smallskip}
\hline
\end{tabular}
\end{table}

\emph{Compressed suffix trees} (\CST) \cite{Sadakane2007} are compressed text
indexes supporting the full functionality of a suffix tree (see
Table~\ref{table:cst operations}). They combine a compressed suffix array, a
compressed representation of the \LCP{} array, and a compressed representation
of suffix tree topology. For the \LCP{} array, there are several common
representations:
\begin{itemize}
\item \LCPbyte{} \cite{Abouelhoda2004} stores the \LCP{} array as a byte
array. If $\mLCP[i] < 255$, the \LCP{} value is stored in the byte array.
Larger values are marked with a $255$ in the byte array and stored separately.
As many texts produce small \LCP{} values, \LCPbyte{} usually requires
$n$ to $1.5n$ bytes of space.
\item We can store the \LCP{} array by using variable-length codes. \LCPdac{}
uses \emph{directly addressable codes} \cite{Brisaboa2009} for the purpose,
resulting in a structure that is typically somewhat smaller and somewhat
slower than \LCPbyte.
\item The \emph{permuted} \LCP{} (\PLCP) \emph{array} \cite{Sadakane2007}
$\mPLCP[1,n]$ is the \LCP{} array stored in text order and used as $\mLCP[i] =
\mPLCP[\mSA[i]]$. Because $\mPLCP[i+1] \ge \mPLCP[i]-1$, the array can be
stored as a bitvector of length $2n$ in $2n+\oh(n)$ bits. If the text is
repetitive, run-length encoding can be used to compress the bitvector to take
even less space \cite{Fischer2009a}. Because accessing \PLCP{} uses \locate,
it is much slower than the above two encodings.
\end{itemize}

Suffix tree topology representations are the main difference between the
various \CST{} proposals. While the compressed suffix arrays and the \LCP{} arrays
are interchangeable, the tree representation determines how various suffix tree
operations are implemented. There are three main families of compressed suffix
trees:
\begin{itemize}
\item \emph{Sadakane's compressed suffix tree} (\CSTsada) \cite{Sadakane2007}
uses a \emph{balanced parentheses} representation for the tree. Each node is
encoded as an opening parenthesis, followed by the encodings of its children
and a closing parenthesis. This can be encoded as a bitvector of length $2n'$,
where $n'$ is the number of nodes, requiring up to $4n+\oh(n)$ bits.
\CSTsada{} tends to be larger and faster than the other compressed suffix
trees \cite{Gog2011a,Abeliuk2013}.
\item The \emph{fully compressed suffix tree} (\FCST) of Russo et
al.~\cite{Russo2011,Navarro2014a} aims to use as little space as possible. It
does not require an \LCP{} array at all, and stores a balanced parentheses
representation for a sampled subset of suffix tree nodes in $\oh(n)$ bits.
Unsampled nodes are retrieved by following suffix links. \FCST{} is smaller
and much slower than the other compressed suffix trees
\cite{Russo2011,Abeliuk2013}.
\item Fischer et al.~\cite{Fischer2009a} proposed an intermediate
representation, \CSTnpr, based on lcp-intervals. Tree navigation is handled
by searching for the values defining the lcp-intervals. \emph{Range minimum
queries} $\mrmq(sp,ep)$ find the leftmost minimal value in $\mLCP[sp,ep]$,
while \emph{next/previous smaller value} queries $\mnsv(i)$/$\mpsv(i)$ find
the next/previous \LCP{} value smaller than $\mLCP[i]$. After the improvements
by various authors \cite{Ohlebusch2009,Ohlebusch2010,Gog2011a,Abeliuk2013},
the \CSTnpr{} is perhaps the most practical compressed suffix tree.
\end{itemize}

For typical texts and component choices, the size of compressed suffix trees
ranges from the $1.5n$ to $3n$ bytes of \CSTsada{} to the $0.5n$ to $n$ bytes
of \FCST{} \cite{Gog2011a,Abeliuk2013}. There are also some \CST{} variants
for repetitive texts, such as versioned document collections and collections
of individual genomes. Abeliuk et al.~\cite{Abeliuk2013} developed a variant
of \CSTnpr{} that can sometimes be smaller than $n$ bits, while achieving
similar performance as the \FCST. Navarro and Ordóñez \cite{Navarro2014} used
grammar-based compression for the tree representation of \CSTsada. The
resulting compressed suffix tree (\GCT) requires slightly more space than the
\CSTnpr{} of Abeliuk et al., while being closer to the non-repetitive
\CSTsada{} and \CSTnpr{} in performance.

\subsection{Relative Lempel-Ziv}\label{sect:rlz}

\emph{Relative Lempel-Ziv} (\RLZ) parsing \cite{Kuruppu2010} compresses
\emph{target} sequence $S$ relative to \emph{reference} sequence $R$. The
target sequence is represented as a concatenation of $z$ \emph{phrases} $w_{i}
= (p_{i}, \ell_{i}, c_{i})$, where $p_{i}$ is the starting position of the
phrase in the reference, $\ell_{i}$ is the length of the copied substring, and
$c_{i}$ is the \emph{mismatch} character. If phrase $w_{i}$ starts from
position $p'$ in the target, then $S[p',p'+\ell_{i}-1] =
R[p_{i},p_{i}+\ell_{i}-1]$ and $S[p'+\ell_{i}] = c_{i}$.

The shortest \RLZ{} parsing of the target sequence can be found in
(essentially) linear time. The algorithm builds a \CSA{} for the reverse of
the reference sequence, and then parses the target sequence greedily by using
backward searching. If the edit distance between the reference and the target
is $s$, we need at most $s$ phrases to represent the target sequence. On the
other hand, because the relative order of the phrases can be different in
sequences $R$ and $S$, the edit distance can be much larger than the number of
phrases in the shortest \RLZ{} parsing.

In a straightforward implementation, the \emph{phrase pointers} $p_{i}$ and
the mismatch characters $c_{i}$ can be stored in arrays $W_{p}$ and
$W_{c}$. These arrays take $z \log \abs{R}$ bits and $z \log \sigma$ bits,
respectively. To support random access to the target sequence, we can encode
phrase lengths as a bitvector $W_{\ell}$ of length $\abs{S}$ \cite{Kuruppu2010}.
We set $W_{\ell}[j] = 1$, if $S[j]$ is the first character of a phrase. The
bitvector requires $z \log \frac{n}{z} + \Oh(z)$ bits, if we use the
\sdarray{} representation \cite{Okanohara2007}. To extract $S[j]$, we first
determine the phrase $w_{i}$, with $i = \mrank_{1}(W_{\ell}, j)$. If
$W_{\ell}[j+1] = 1$, we return the mismatch character $W_{c}[i]$. Otherwise
we determine the phrase offset with a \select{} query, and return the character
$R[W_{p}[i] + j - \mselect_{1}(W_{\ell}, i)]$.

The \select{} query can be avoided by using \emph{relative pointers} instead
of absolute pointers \cite{Ferrada2014}. By setting $W_{p}[i] = p_{i} -
\mselect_{1}(W_{\ell}, i)$, the general case simplifies to $S[j] = R[W_{p}[i]
+ j]$. If most of the differences between the reference and the target
sequence are single-character \emph{substitutions}, $p_{i+1}$ will often be
$p_{i} + \ell_{i} + 1$. This corresponds to $W_{p}[i+1] = W_{p}[i]$ with
relative pointers, making \emph{run-length encoding} the pointer array
worthwhile.


